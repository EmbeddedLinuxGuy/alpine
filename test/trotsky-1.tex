
\documentclass[12pt]{article}
\title{Terrorism and Communism}
\usepackage[bmargin=.25in, lmargin=.25in, rmargin=.25in, tmargin=.25in, paperwidth=4.25in, paperheight=5.5in]{geometry}
\usepackage{graphicx}
\begin{document}
\thispagestyle{empty}
\begin{center}
\textbf{\Large Terrorism and Communism}\\
\textbf{Leon Trotsky}\\

\end{center}

\begin{center}
\textbf{\small I. \hspace{1em} THE BALANCE OF POWER
}
\end{center}
\begin{center}
\begin{verbatim}
                              /  
                   __       //   
                   -\= \=\ //    
                 --=_\=---//=--  
               -_==/  \/ //\/--  
                ==/   /O   O\==--
   _ _ _ _     /_/    \  ]  /--  
  /\ ( (- \    /       ] ] ]==-  
 (\ _\_\_\-\__/     \  (,_,)--   
(\_/                 \     \-    
\/      /       (   ( \  ] /)
/      (         \   \_ \./ )    
(       \         \      )  \    
(       /\_ _ _ _ /---/ /\_  \   
 \     / \     / ____/ /   \  \  
  (   /   )   / /  /__ )   (  )  
  (  )   / __/ '---`       / /   
  \  /   \ \             _/ /   
  ] ]     )_\_         /__\/     
  /_\     ]___\                  
 (___)                           
\end{verbatim}
\end{center}
\pagebreak


\vspace{12pt}
The argument which is repeated again and again in criticisms of the
Soviet system in Russia, and particularly in criticisms of
revolutionary attempts to set up a similar structure in other
countries, is the argument based on the balance of power. The Soviet
regime in Russia is utopian -- ``because it does not correspond to the
balance of power." Backward Russia cannot put objects before itself
which would be appropriate to advanced Germany. And for the
proletariat of Germany it would be madness to take political power
into its own hands, as this ``at the present moment" would disturb the
balance of power. The League of Nations is imperfect, but still
corresponds to the balance of power. The struggle for the overthrow of
imperialist supremacy is utopian -- the balance of power only requires a
revision of the Versailles Treaty. When Longuet hobbled after Wilson
this took place, not because of the political decomposition of
Longuet, but in honor of the law of the balance of power. The Austrian
president, Seitz, and the chancellor, Renner, must, in the opinion of
Friedrich Adler, exercise their bourgeois impotence at the central
posts of the bourgeois republic, for otherwise the balance of power
would be infringed. Two years before the world war, Karl Renner, then
not a chancellor, but a ``Marxist" advocate of opportunism, explained
to me that the regime of June 3 -- that is, the union of landlords and
capitalists crowned by the monarchy -- must inevitably maintain itself
in Russia during a whole historical period, as it answered to the
balance of power.

\vspace{12pt}
What is this balance of power after all -- that sacramental formula
which is to define, direct, and explain the whole course of history,
wholesale and retail? Why exactly is it that the formula of the
balance of power, in the mouth of Kautsky and his present school,
inevitably appears as a justification of indecision, stagnation,
cowardice and treachery?

\vspace{12pt}
By the balance of power they understand everything you please: the
level of production attained, the degree of differentiation of
classes, the number of organized workers, the total funds at the
disposal of the trade unions, sometimes the results of the last
parliamentary elections, frequently the degree of readiness for
compromise on the part of the ministry, or the degree of effrontery of
the financial oligarchy. Most frequently, it means that summary
political impression which exists in the mind of a half-blind pedant,
or a so-called realist politician, who, though he has absorbed the
phraseology of Marxism, in reality is guided by the most shallow
manoeuvres, bourgeois prejudices, and parliamentary ``tactics." After
a whispered conversation with the director of the police department,
an Austrian Social-Democratic politician in the good, and not so far
off, old times always knew exactly whether the balance of power
permitted a peaceful street demonstration in Vienna on May Day. In the
case of the Eberts, Scheidemanns and Davids, the balance of power was,
not so very long ago, calculated exactly by the number of fingers
which were extended to them at their meeting in the Reichstag with
Bethmann-Hollweg, or with Ludendorff himself.

\vspace{12pt}
According to Friedrich Adler, the establishment of a Soviet
dictatorship in Austria would be a fatal infraction of the balance of
power; the Entente would condemn Austria to starvation. In proof of
this, Friedrich Adler, at the July congress of Soviets, pointed to
Hungary, where at that time the Hungarian Renners had not yet, with
the help of the Hungarian Adlers, overthrown the dictatorship of the
Soviets. At the first glance, it might really seem that Friedrich
Adler was right in the case of Hungary. The proletarian dictatorship
was overthrown there soon afterwards, and its place was filled by the
ministry of the reactionary Friedrich. But it is quite justifiable to
ask: Did the latter correspond to the balance of power? At all events,
Friedrich and his Huszar might not even temporarily have seized power
had it not been for the Roumanian army. Hence, it is clear that, when
discussing the fate of the Soviet Government in Hungary, it is
necessary to take account of the ``balance of power," at all events in
two countries -- in Hungary itself, and in its neighbor, Roumania. But
it is not difficult to grasp that we cannot stop at this. If the
dictatorship of the Soviets had been set up in Austria before the
maturing of the Hungarian crisis, the overthrow of the Soviet regime
in Budapest would have been an infinitely more difficult task.
Consequently, we have to include Austria also, together with the
treacherous policy of Friedrich Adler, in that balance of power which
determined the temporary fall of the Soviet Government in Hungary.

\vspace{12pt}
Friedrich Adler himself, however, seeks the key to the balance of
power, not in Russia and Hungary, but in the West, in the countries of
Clemenceau and Lloyd George. They have in their hands bread and
coal -- and really bread and coal, especially in our time, are just as
foremost factors in the mechanism of the balance of power as cannon in
the constitution of Lassalle. Brought down from the heights, Adler's
idea consists, consequently, in this: that the Austrian proletariat
must not seize power until such time, as it is permitted to do so by
Clemenceau (or Millerand -- \emph{i.e.}, a Clemenceau of the second
order).

\vspace{12pt}
However, even here it is permissible to ask: Does the policy of
Clemenceau himself really correspond to the balance of power? At the
first glance it may appear that it corresponds well enough, and, if
it cannot be proved, it is, at least, guaranteed by Clemenceau's
gendarmes, who break up working-class meetings, and arrest and
shoot Communists. But here we cannot but remember that the
terrorist measures of the Soviet Government -- that is, the same
searches, arrests, and executions, only directed against the
counter-revolutionaries -- are considered by some people as a proof that
the Soviet Government does \emph{not} correspond to the balance of power.
In vain would we, however, begin to seek in our time, anywhere in the
world, a regime which, to preserve itself, did not have recourse to
measures of stern mass repression. This means that hostile class
forces, having broken through the framework of every kind of
law -- including that of ``democracy" -- are striving to find their new
balance by means of a merciless struggle.

\vspace{12pt}
When the Soviet system was being instituted in Russia, not only the
capitalist politicians, but also the Socialist opportunists of all
countries proclaimed it an insolent challenge to the balance of
forces. On this score, there was no quarrel between Kautsky, the
Austrian Count Czernin, and the Bulgarian Premier, Radoslavov. Since
that time, the Austro-Hungarian and German monarchies have collapsed,
and the most powerful militarism in the world has fallen into dust.
The Soviet regime has held out. The victorious countries of the
Entente have mobilized and hurled against it all they could. The
Soviet Government has stood firm. Had Kautsky, Friedrich Adler, and
Otto Bauer been told that the system of the dictatorship of the
proletariat would hold out in Russia -- first against the attack of
German militarism, and then in a ceaseless war with the militarism of
the Entente countries -- the sages of the Second International would
have considered such a prophecy a laughable misunderstanding of the
``balance of power."

\vspace{12pt}
The balance of political power at any given moment is determined under
the influence of fundamental and secondary factors of differing
degrees of effectiveness, and only in its most fundamental quality is
it determined by the stage of the development of production. The
social structure of a people is extraordinarily behind the development
of its productive forces. The lower middle-classes, and particularly
the peasantry, retain their existence long after their economic
methods have been made obsolete, and have been condemned, by the
technical development of the productive powers of society. The
consciousness of the masses, in its turn, is extraordinarily behind
the development of their social relations, the consciousness of the
old Socialist parties is a whole epoch behind the state of mind of the
masses, and the consciousness of the old parliamentary and trade union
leaders, more reactionary than the consciousness of their party,
represents a petrified mass which history has been unable hitherto
either to digest or reject. In the parliamentary epoch, during the
period of stability of social relations, the psychological
factor -- without great error -- was the foundation upon which all current
calculations were based. It was considered that parliamentary
elections reflected the balance of power with sufficient exactness.
The imperialist war, which upset all bourgeois society, displayed the
complete uselessness of the old criteria. The latter completely
ignored those profound historical factors which had gradually been
accumulating in the preceding period, and have now, all at once,
appeared on the surface, and have begun to determine the course of
history.

\vspace{12pt}
The political worshippers of routine, incapable of surveying the
historical process in its complexity, in its internal clashes and
contradictions, imagined to themselves that history was preparing the
way for the Socialist order simultaneously and systematically on all
sides, so that concentration of production and the development of a
Communist morality in the producer and the consumer mature
simultaneously with the electric plough and a parliamentary majority.
Hence the purely mechanical attitude towards parliamentarism, which,
in the eyes of the majority of the statesmen of the Second
International, indicated the degree to which society was prepared for
Socialism as accurately as the manometer indicates the pressure of
steam. Yet there is nothing more senseless than this mechanized
representation of the development of social relations.

\vspace{12pt}
If, beginning with the productive bases of society, we ascend the
stages of the superstructure -- classes, the State, laws, parties, and
so on -- it may be established that the weight of each additional part
of the superstructure is not simply to be added to, but in many cases
to be multiplied by, the weight of all the preceding stages. As a
result, the political consciousness of groups which long imagined
themselves to be among the most advanced, displays itself, at a moment
of change, as a colossal obstacle in the path of historical
development. To-day it is quite beyond doubt that the parties of the
Second International, standing at the head of the proletariat, which
dared not, could not, and would not take power into their hands at the
most critical moment of human history, and which led the proletariat
along the road of mutual destruction in the interests of imperialism,
proved a \emph{decisive factor} of the counter-revolution.

\vspace{12pt}
The great forces of production -- that shock factor in historical
development -- were choked in those obsolete institutions of the
superstructure (private property and the national State) in which they
found themselves locked by all preceding development. Engendered by
capitalism, the forces of production were knocking at all the walls of
the bourgeois national State, demanding their emancipation by means of
the Socialist organization of economic life on a world scale. The
stagnation of social groupings, the stagnation of political forces,
which proved themselves incapable of destroying the old class
groupings, the stagnation, stupidity and treachery of the directing
Socialist parties, which had assumed to themselves in reality the
defense of bourgeois society -- all these factors led to an elemental
revolt of the forces of production, in the shape of the imperialist
war. Human technical skill, the most revolutionary factor in history,
arose with the might accumulated during scores of years against the
disgusting conservatism and criminal stupidity of the Scheidemanns,
Kautskies, Renaudels, Vanderveldes and Longuets, and, by means of its
howitzers, machine-guns, dreadnoughts and aeroplanes, it began a
furious pogrom of human culture.

\vspace{12pt}
In this way the cause of the misfortunes at present experienced by
humanity is precisely that the development of the technical command of
men over nature has \emph{long ago} grown ripe for the socialization
of economic life. The proletariat has occupied a place in production
which completely guarantees its dictatorship, while the most
intelligent forces in history -- the parties and their leaders -- have
been discovered to be still wholly under the yoke of the old
prejudices, and only fostered a lack of faith among the masses in
their own power. In quite recent years Kautsky used to understand
this. ``The proletariat at the present time has grown so strong," wrote
Kautsky in his pamphlet, \emph{The Path to Power}, ``that it can calmly
await the coming war. There can be no more talk of a \emph{premature
revolution}, now that the proletariat has drawn from the present
structure of the State such strength as could be drawn therefrom, and
now that its reconstruction has become a condition of the
proletariat's further progress." From the moment that the development
of productive forces, outgrowing the framework of the bourgeois
national State, drew mankind into an epoch of crises and convulsions,
the consciousness of the masses was shaken by dread shocks out of the
comparative equilibrium of the preceding epoch. The routine and
stagnation of its mode of living, the hypnotic suggestion of peaceful
legality, had already ceased to dominate the proletariat. But it had
not yet stepped, consciously and courageously, on to the path of open
revolutionary struggle. It wavered, passing through the last moment of
unstable equilibrium. At such a moment of psychological change, the
part played by the summit -- the State, on the one hand, and the
revolutionary Party on the other -- acquires a colossal importance. A
determined push from left or right is sufficient to move the
proletariat, for a certain period, to one or the other side. We saw
this in 1914, when, under the united pressure of imperialist
governments and Socialist patriotic parties, the working class was all
at once thrown out of its equilibrium and hurled on to the path of
imperialism. We have since seen how the experience of the war, the
contrasts between its results and its first objects, is shaking the
masses in a revolutionary sense, making them more and more capable of
an open revolt against capitalism. In such conditions, the presence of
a revolutionary party, which renders to itself a clear account of the
motive forces of the present epoch, and understands the exceptional
role amongst them of a revolutionary class; which knows its
inexhaustible, but unrevealed, powers; which believes in that class
and believes in itself; which knows the power of revolutionary method
in an epoch of instability of all social relations; which is ready to
employ that method and carry it through to the end -- the presence of
such a party represents a factor of incalculable historical
importance.

\vspace{12pt}
And, on the other hand, the Socialist party, enjoying traditional
influence, which does \emph{not} render itself an account of what is going
on around it, which does \emph{not} understand the revolutionary situation,
and, therefore, finds no key to it, which does \emph{not} believe in either
the proletariat or itself -- such a party in our time is the most
mischievous stumbling block in history, and a source of confusion and
inevitable chaos.

\vspace{12pt}
Such is now the role of Kautsky and his sympathizers. They teach the
proletariat not to believe in itself, but to believe its reflection in
the crooked mirror of democracy which has been shattered by the
jack-boot of militarism into a thousand fragments. The decisive factor
in the revolutionary policy of the working class must be, in their
view, not the international situation, not the actual collapse of
capitalism, not that social collapse which is generated thereby, not
that concrete necessity of the supremacy of the working class for
which the cry arises from the smoking ruins of capitalist
civilization -- not all this must determine the policy of the
revolutionary party of the proletariat -- but that counting of votes
which is carried out by the capitalist tellers of parliamentarism.
Only a few years ago, we repeat, Kautsky seemed to understand the real
inner meaning of the problem of revolution. ``Yes, the proletariat
represents the sole revolutionary class of the nation," wrote Kautsky
in his pamphlet, \emph{The Path to Power}. It follows that every collapse
of the capitalist order, whether it be of a moral, financial, or
military character, implies the bankruptcy of all the bourgeois
parties responsible for it, and signifies that the sole way out of the
blind alley is the establishment of the power of the \emph{proletariat}.
And to-day the party of prostration and cowardice, the party of
Kautsky, says to the working class: ``The question is not whether you
to-day are the sole creative force in history; whether you are capable
of throwing aside that ruling band of robbers into which the
propertied classes have developed; the question is not whether anyone
else can accomplish this task on your behalf; the question is not
whether history allows you any postponement (for the present condition
of bloody chaos threatens to bury you yourself, in the near future,
under the last ruins of capitalism). The problem is for the ruling
imperialist bandits to succeed -- yesterday or to-day -- to deceive,
violate, and swindle public opinion, by collecting 51 per cent.\ of the
votes against your 49. Perish the world, but long live the
parliamentary majority!"






\vfill
\begin{center}
{\fontfamily{phv}\selectfont 
Modern Hacker Library\\books.modernhacker.com}
\end{center}
\end{document}
