
\documentclass[12pt]{article}
\title{Demons}
\usepackage[bmargin=.25in, lmargin=.25in, rmargin=.25in, tmargin=.25in, paperwidth=4.25in, paperheight=5.5in]{geometry}
\usepackage{graphicx}
\begin{document}
\thispagestyle{empty}
\begin{center}
\textbf{\Large Demons}\\
\textbf{Fyodor Dostoevsky}\\

\end{center}

\begin{center}
\textbf{\small PART 1\\ CHAPTER I. INTRODUCTORY}\\
\end{center}
\begin{center}
\includegraphics[height=3.7in]{demons-1.jpg}
\end{center}
\hyphenation{STE-PAN TROF-IM-OV-ITCH VER-HOV-EN-SKY}
{\scriptsize 

\vspace{12pt}
     ``Strike me dead, the track has vanished,

     Well, what now? We've lost the way,

     Demons have bewitched our horses,

     Led us in the wilds astray.



\vspace{12pt}
     ``What a number! Whither drift they?

     What's the mournful dirge they sing?

     Do they hail a witch's marriage

     Or a goblin's burying?"



\vspace{12pt}
     A. Pushkin.





\vspace{12pt}
     ``And there was one herd of many swine feeding on this

     mountain; and they besought him that he would suffer them to

     enter into them. And he suffered them.



\vspace{12pt}
     ``Then went the devils out of the man and entered into the

     swine; and the herd ran violently down a steep place into

     the lake and were choked.



\vspace{12pt}
     ``When they that fed them saw what was done, they fled, and

     went and told it in the city and in the country.



\vspace{12pt}
     ``Then they went out to see what was done; and came to Jesus

     and found the man, out of whom the devils were departed,

     sitting at the feet of Jesus, clothed and in his right mind;

     and they were afraid."



\vspace{12pt}
     Luke, ch. viii. 32-37.









}
\setcounter{page}{1}

{\scriptsize \noindent 

\vspace{12pt}
SOME DETAILS OF THE BIOGRAPHY OF THAT HIGHLY RESPECTED GENTLEMAN STEPAN
TROFIMOVITCH VERHOVENSKY.}
\vspace{12pt}



\vspace{12pt}
IN UNDERTAKING to describe the recent and strange incidents in our town,
till lately wrapped in uneventful obscurity, I find myself forced in
absence of literary skill to begin my story rather far back, that is
to say, with certain biographical details concerning that talented and
highly-esteemed gentleman, Stepan Trofimovitch Verhovensky. I trust that
these details may at least serve as an introduction, while my projected
story itself will come later.


\vspace{12pt}
I will say at once that Stepan Trofimovitch had always filled a
particular role among us, that of the progressive patriot, so to say,
and he was passionately fond of playing the part--so much so that I
really believe he could not have existed without it. Not that I would
put him on a level with an actor at a theatre, God forbid, for I really
have a respect for him. This may all have been the effect of habit, or
rather, more exactly of a generous propensity he had from his earliest
years for indulging in an agreeable day-dream in which he figured as
a picturesque public character. He fondly loved, for instance, his
position as a `persecuted' man and, so to speak, an ``exile." There is a
sort of traditional glamour about those two little words that fascinated
him once for all and, exalting him gradually in his own opinion, raised
him in the course of years to a lofty pedestal very gratifying to
vanity. In an English satire of the last century, Gulliver, returning
from the land of the Lilliputians where the people were only three or
four inches high, had grown so accustomed to consider himself a giant
among them, that as he walked along the streets of London he could not
help crying out to carriages and passers-by to be careful and get out of
his way for fear he should crush them, imagining that they were little
and he was still a giant. He was laughed at and abused for it, and rough
coachmen even lashed at the giant with their whips. But was that just?
What may not be done by habit? Habit had brought Stepan Trofimovitch
almost to the same position, but in a more innocent and inoffensive
form, if one may use such expressions, for he was a most excellent man.


\vspace{12pt}
I am even inclined to suppose that towards the end he had been entirely
forgotten everywhere; but still it cannot be said that his name had
never been known. It is beyond question that he had at one time belonged
to a certain distinguished constellation of celebrated leaders of
the last generation, and at one time--though only for the briefest
moment--his name was pronounced by many hasty persons of that day almost
as though it were on a level with the names of Tchaadaev, of Byelinsky,
of Granovsky, and of Herzen, who had only just begun to write abroad.
But Stepan Trofimovitch's activity ceased almost at the moment it began,
owing, so to say, to a ``vortex of combined circumstances." And would you
believe it? It turned out afterwards that there had been no `vortex' and
even no `circumstances,' at least in that connection. I only learned
the other day to my intense amazement, though on the most unimpeachable
authority, that Stepan Trofimovitch had lived among us in our province
not as an `exile' as we were accustomed to believe, and had never even
been under police supervision at all. Such is the force of imagination!
All his life he sincerely believed that in certain spheres he was a
constant cause of apprehension, that every step he took was watched
and noted, and that each one of the three governors who succeeded one
another during twenty years in our province came with special and uneasy
ideas concerning him, which had, by higher powers, been impressed upon
each before everything else, on receiving the appointment. Had anyone
assured the honest man on the most irrefutable grounds that he had
nothing to be afraid of, he would certainly have been offended. Yet
Stepan Trofimovitch was a most intelligent and gifted man, even, so to
say, a man of science, though indeed, in science... well, in fact he
had not done such great things in science. I believe indeed he had done
nothing at all. But that's very often the case, of course, with men of
science among us in Russia.


\vspace{12pt}
He came back from abroad and was brilliant in the capacity of lecturer
at the university, towards the end of the forties. He only had time
to deliver a few lectures, I believe they were about the Arabs; he
maintained, too, a brilliant thesis on the political and Hanseatic
importance of the German town Hanau, of which there was promise in the
epoch between 1413 and 1428, and on the special and obscure reasons
why that promise was never fulfilled. This dissertation was a cruel
and skilful thrust at the Slavophils of the day, and at once made him
numerous and irreconcilable enemies among them. Later on--after he had
lost his post as lecturer, however--he published (by way of revenge,
so to say, and to show them what a man they had lost) in a progressive
monthly review, which translated Dickens and advocated the views of
George Sand, the beginning of a very profound investigation into the
causes, I believe, of the extraordinary moral nobility of certain
knights at a certain epoch or something of that nature.


\vspace{12pt}
Some lofty and exceptionally noble idea was maintained in it, anyway.
It was said afterwards that the continuation was hurriedly forbidden and
even that the progressive review had to suffer for having printed the
first part. That may very well have been so, for what was not possible
in those days? Though, in this case, it is more likely that there
was nothing of the kind, and that the author himself was too lazy to
conclude his essay. He cut short his lectures on the Arabs because,
somehow and by some one (probably one of his reactionary enemies) a
letter had been seized giving an account of certain circumstances, in
consequence of which some one had demanded an explanation from him. I
don't know whether the story is true, but it was asserted that at the
same time there was discovered in Petersburg a vast, unnatural, and
illegal conspiracy of thirty people which almost shook society to its
foundations. It was said that they were positively on the point of
translating Fourier. As though of design a poem of Stepan Trofimovitch's
was seized in Moscow at that very time, though it had been written six
years before in Berlin in his earliest youth, and manuscript copies had
been passed round a circle consisting of two poetical amateurs and one
student. This poem is lying now on my table. No longer ago than last
year I received a recent copy in his own handwriting from Stepan
Trofimovitch himself, signed by him, and bound in a splendid red leather
binding. It is not without poetic merit, however, and even a certain
talent. It's strange, but in those days (or to be more exact, in the
thirties) people were constantly composing in that style. I find it
difficult to describe the subject, for I really do not understand it.
It is some sort of an allegory in lyrical-dramatic form, recalling the
second part of Faust. The scene opens with a chorus of women, followed
by a chorus of men, then a chorus of incorporeal powers of some sort,
and at the end of all a chorus of spirits not yet living but very
eager to come to life. All these choruses sing about something very
indefinite, for the most part about somebody's curse, but with a tinge
of the higher humour. But the scene is suddenly changed. There begins a
sort of `festival of life' at which even insects sing, a tortoise
comes on the scene with certain sacramental Latin words, and even, if
I remember aright, a mineral sings about something that is a quite
inanimate object. In fact, they all sing continually, or if they
converse, it is simply to abuse one another vaguely, but again with
a tinge of higher meaning. At last the scene is changed again; a
wilderness appears, and among the rocks there wanders a civilized young
man who picks and sucks certain herbs. Asked by a fairy why he sucks
these herbs, he answers that, conscious of a superfluity of life in
himself, he seeks forgetfulness, and finds it in the juice of these
herbs, but that his great desire is to lose his reason at once (a desire
possibly superfluous). Then a youth of indescribable beauty rides in on
a black steed, and an immense multitude of all nations follow him.
The youth represents death, for whom all the peoples are yearning. And
finally, in the last scene we are suddenly shown the Tower of Babel, and
certain athletes at last finish building it with a song of new hope, and
when at length they complete the topmost pinnacle, the lord (of Olympia,
let us say) takes flight in a comic fashion, and man, grasping the
situation and seizing his place, at once begins a new life with new
insight into things. Well, this poem was thought at that time to be
dangerous. Last year I proposed to Stepan Trofimovitch to publish it,
on the ground of its perfect harmlessness nowadays, but he declined
the suggestion with evident dissatisfaction. My view of its complete
\begin{figure}[!ht]
\begin{center}
\includegraphics[height=4.1in]{demons-2.jpg}
\end{center}
\end{figure}
harmlessness evidently displeased him, and I even ascribe to it a
certain coldness on his part, which lasted two whole months.


\vspace{12pt}
And what do you think? Suddenly, almost at the time I proposed printing
it here, our poem was published abroad in a collection of revolutionary
verse, without the knowledge of Stepan Trofimovitch. He was at
first alarmed, rushed to the governor, and wrote a noble letter in
self-defence to Petersburg. He read it to me twice, but did not send
it, not knowing to whom to address it. In fact he was in a state of
agitation for a whole month, but I am convinced that in the secret
recesses of his heart he was enormously flattered. He almost took the
copy of the collection to bed with him, and kept it hidden under his
mattress in the daytime; he positively would not allow the women to turn
his bed, and although he expected every day a telegram, he held his head
high. No telegram came. Then he made friends with me again, which is a
proof of the extreme kindness of his gentle and unresentful heart.



\vspace{12pt}
II


\vspace{12pt}
Of course I don't assert that he had never suffered for his convictions
at all, but I am fully convinced that he might have gone on lecturing
on his Arabs as long as he liked, if he had only given the necessary
explanations. But he was too lofty, and he proceeded with peculiar haste
to assure himself that his career was ruined for ever ``by the vortex of
circumstance." And if the whole truth is to be told the real cause of
the change in his career was the very delicate proposition which had
been made before and was then renewed by Varvara Petrovna Stavrogin, a
lady of great wealth, the wife of a lieutenant-general, that he should
undertake the education and the whole intellectual development of her
only son in the capacity of a superior sort of teacher and friend, to
say nothing of a magnificent salary. This proposal had been made to
him the first time in Berlin, at the moment when he was first left a
widower. His first wife was a frivolous girl from our province, whom he
married in his early and unthinking youth, and apparently he had had a
great deal of trouble with this young person, charming as she was,
owing to the lack of means for her support; and also from other, more
delicate, reasons. She died in Paris after three years' separation
from him, leaving him a son of five years old; ``the fruit of our first,
joyous, and unclouded love," were the words the sorrowing father once
let fall in my presence.


\vspace{12pt}
The child had, from the first, been sent back to Russia, where he was
brought up in the charge of distant cousins in some remote region.
Stepan Trofimovitch had declined Varvara Petrovna's proposal on that
occasion and had quickly married again, before the year was over, a
taciturn Berlin girl, and, what makes it more strange, there was no
particular necessity for him to do so. But apart from his marriage there
were, it appears, other reasons for his declining the situation. He was
tempted by the resounding fame of a professor, celebrated at that time,
and he, in his turn, hastened to the lecturer's chair for which he had
been preparing himself, to try his eagle wings in flight. But now with
singed wings he naturally remembered the proposition which even then had
made him hesitate. The sudden death of his second wife, who did not live
a year with him, settled the matter decisively. To put it plainly it was
all brought about by the passionate sympathy and priceless, so to
speak, classic friendship of Varvara Petrovna, if one may use such
an expression of friendship. He flung himself into the arms of this
friendship, and his position was settled for more than twenty years. I
use the expression `flung himself into the arms of,' but God forbid that
anyone should fly to idle and superfluous conclusions. These embraces
must be understood only in the most loftily moral sense. The most
refined and delicate tie united these two beings, both so remarkable,
for ever.


\vspace{12pt}
The post of tutor was the more readily accepted too, as the property--a
very small one--left to Stepan Trofimovitch by his first wife was close
to Skvoreshniki, the Stavrogins' magnificent estate on the outskirts of
our provincial town. Besides, in the stillness of his study, far from
the immense burden of university work, it was always possible to devote
himself to the service of science, and to enrich the literature of his
country with erudite studies. These works did not appear. But on the
other hand it did appear possible to spend the rest of his life, more
than twenty years, `a reproach incarnate,' so to speak, to his native
country, in the words of a popular poet:


\vspace{12pt}
\emph{Reproach incarnate thou didst stand}


\emph{Erect before thy Fatherland,}


\emph{O Liberal idealist!}




\vspace{12pt}
But the person to whom the popular poet referred may perhaps have had
the right to adopt that pose for the rest of his life if he had wished
to do so, though it must have been tedious. Our Stepan Trofimovitch was,
to tell the truth, only an imitator compared with such people; moreover,
he had grown weary of standing erect and often lay down for a while.
But, to do him justice, the `incarnation of reproach' was preserved even
in the recumbent attitude, the more so as that was quite sufficient for
the province. You should have seen him at our club when he sat down to
cards. His whole figure seemed to exclaim ``Cards! Me sit down to whist
with you! Is it consistent? Who is responsible for it? Who has shattered
my energies and turned them to whist? Ah, perish, Russia!" and he would
majestically trump with a heart.


\vspace{12pt}
And to tell the truth he dearly loved a game of cards, which led him,
especially in later years, into frequent and unpleasant skirmishes with
Varvara Petrovna, particularly as he was always losing. But of that
later. I will only observe that he was a man of tender conscience (that
is, sometimes) and so was often depressed. In the course of his twenty
years' friendship with Varvara Petrovna he used regularly, three or
four times a year, to sink into a state of `patriotic grief,' as it
was called among us, or rather really into an attack of spleen, but our
estimable Varvara Petrovna preferred the former phrase. Of late years
his grief had begun to be not only patriotic, but at times alcoholic
too; but Varvara Petrovna's alertness succeeded in keeping him all his
life from trivial inclinations. And he needed some one to look after him
indeed, for he sometimes behaved very oddly: in the midst of his exalted
sorrow he would begin laughing like any simple peasant. There were
moments when he began to take a humorous tone even about himself. But
there was nothing Varvara Petrovna dreaded so much as a humorous tone.
She was a woman of the classic type, a female Mæcenas, invariably
guided only by the highest considerations. The influence of this exalted
lady over her poor friend for twenty years is a fact of the first
importance. I shall need to speak of her more particularly, which I now
proceed to do.



\vspace{12pt}
III


\vspace{12pt}
There are strange friendships. The two friends are always ready to fly
at one another, and go on like that all their lives, and yet they cannot
separate. Parting, in fact, is utterly impossible. The one who has begun
the quarrel and separated will be the first to fall ill and even die,
perhaps, if the separation comes off. I know for a positive fact that
several times Stepan Trofimovitch has jumped up from the sofa and
beaten the wall with his fists after the most intimate and emotional
\begin{figure}[!ht]
\begin{center}
\includegraphics[height=4.1in]{demons-3.jpg}
\end{center}
\end{figure}
\emph{t\^ete-`a-t\^ete} with Varvara Petrovna.


\vspace{12pt}
This proceeding was by no means an empty symbol; indeed, on one
occasion, he broke some plaster off the wall. It may be asked how I come
to know such delicate details. What if I were myself a witness of it?
What if Stepan Trofimovitch himself has, on more than one occasion,
sobbed on my shoulder while he described to me in lurid colours all his
most secret feelings. (And what was there he did not say at such times!)
But what almost always happened after these tearful outbreaks was that
next day he was ready to crucify himself for his ingratitude. He would
send for me in a hurry or run over to see me simply to assure me that
Varvara Petrovna was ``an angel of honour and delicacy, while he was very
much the opposite." He did not only run to confide in me, but, on more
than one occasion, described it all to her in the most eloquent letter,
and wrote a full signed confession that no longer ago than the day
before he had told an outsider that she kept him out of vanity, that
she was envious of his talents and erudition, that she hated him and was
only afraid to express her hatred openly, dreading that he would leave
her and so damage her literary reputation, that this drove him to
self-contempt, and he was resolved to die a violent death, and that he
was waiting for the final word from her which would decide everything,
and so on and so on in the same style. You can fancy after this what
an hysterical pitch the nervous outbreaks of this most innocent of
all fifty-year-old infants sometimes reached! I once read one of these
letters after some quarrel between them, arising from a trivial matter,
but growing venomous as it went on. I was horrified and besought him not
to send it.


\vspace{12pt}
``I must... more honourable... duty... I shall die if I don't confess
everything, everything!" he answered almost in delirium, and he did send
the letter.


\vspace{12pt}
That was the difference between them, that Varvara Petrovna never would
have sent such a letter. It is true that he was passionately fond of
writing, he wrote to her though he lived in the same house, and during
hysterical interludes he would write two letters a day. I know for a
fact that she always read these letters with the greatest attention,
even when she received two a day, and after reading them she put them
away in a special drawer, sorted and annotated; moreover, she pondered
them in her heart. But she kept her friend all day without an answer,
met him as though there were nothing the matter, exactly as though
nothing special had happened the day before. By degrees she broke him in
so completely that at last he did not himself dare to allude to what had
happened the day before, and only glanced into her eyes at times. But
she never forgot anything, while he sometimes forgot too quickly, and
encouraged by her composure he would not infrequently, if friends came
in, laugh and make jokes over the champagne the very same day. With what
malignancy she must have looked at him at such moments, while he noticed
nothing! Perhaps in a week's time, a month's time, or even six months
later, chancing to recall some phrase in such a letter, and then the
whole letter with all its attendant circumstances, he would suddenly
grow hot with shame, and be so upset that he fell ill with one of his
attacks of ``summer cholera.` These attacks of a sort of 'summer cholera"
were, in some cases, the regular consequence of his nervous agitations
and were an interesting peculiarity of his physical constitution.


\vspace{12pt}
No doubt Varvara Petrovna did very often hate him. But there was one
thing he had not discerned up to the end: that was that he had become
for her a son, her creation, even, one may say, her invention; he had
become flesh of her flesh, and she kept and supported him not simply
from ``envy of his talents." And how wounded she must have been by such
suppositions! An inexhaustible love for him lay concealed in her heart
in the midst of continual hatred, jealousy, and contempt. She would not
let a speck of dust fall upon him, coddled him up for twenty-two years,
would not have slept for nights together if there were the faintest
breath against his reputation as a poet, a learned man, and a public
character. She had invented him, and had been the first to believe in
her own invention. He was, after a fashion, her day-dream.... But in
return she exacted a great deal from him, sometimes even slavishness. It
was incredible how long she harboured resentment. I have two anecdotes
to tell about that.



\vspace{12pt}
IV


\vspace{12pt}
On one occasion, just at the time when the first rumours of the
emancipation of the serfs were in the air, when all Russia was exulting
and making ready for a complete regeneration, Varvara Petrovna was
visited by a baron from Petersburg, a man of the highest connections,
and very closely associated with the new reform. Varvara Petrovna prized
such visits highly, as her connections in higher circles had grown
weaker and weaker since the death of her husband, and had at last ceased
altogether. The baron spent an hour drinking tea with her. There was no
one else present but Stepan Trofimovitch, whom Varvara Petrovna invited
and exhibited. The baron had heard something about him before or
affected to have done so, but paid little attention to him at tea.
Stepan Trofimovitch of course was incapable of making a social blunder,
and his manners were most elegant. Though I believe he was by no means
of exalted origin, yet it happened that he had from earliest childhood
been brought up in a Moscow household--of high rank, and consequently
was well bred. He spoke French like a Parisian. Thus the baron was to
have seen from the first glance the sort of people with whom Varvara
Petrovna surrounded herself, even in provincial seclusion. But things
did not fall out like this. When the baron positively asserted the
absolute truth of the rumours of the great reform, which were then
only just beginning to be heard, Stepan Trofimovitch could not contain
himself, and suddenly shouted ``Hurrah!" and even made some gesticulation
indicative of delight. His ejaculation was not over-loud and quite
polite, his delight was even perhaps premeditated, and his gesture
purposely studied before the looking-glass half an hour before tea. But
something must have been amiss with it, for the baron permitted himself
a faint smile, though he, at once, with extraordinary courtesy, put in
a phrase concerning the universal and befitting emotion of all Russian
hearts in view of the great event. Shortly afterwards he took his
leave and at parting did not forget to hold out two fingers to Stepan
Trofimovitch. On returning to the drawing-room Varvara Petrovna was
at first silent for two or three minutes, and seemed to be looking for
something on the table. Then she turned to Stepan Trofimovitch, and with
pale face and flashing eyes she hissed in a whisper:


\vspace{12pt}
``I shall never forgive you for that!"


\vspace{12pt}
Next day she met her friend as though nothing had happened, she never
referred to the incident, but thirteen years afterwards, at a tragic
moment, she recalled it and reproached him with it, and she turned pale,
just as she had done thirteen years before. Only twice in the course of
her life did she say to him:


\vspace{12pt}
``I shall never forgive you for that!"


\vspace{12pt}
The incident with the baron was the second time, but the first incident
was so characteristic and had so much influence on the fate of Stepan
Trofimovitch that I venture to refer to that too.

\begin{figure}[!ht]
\begin{center}
\includegraphics[height=4.1in]{demons-4.jpg}
\end{center}
\end{figure}
\vspace{12pt}
It was in 1855, in spring-time, in May, just after the news had reached
Skvoreshniki of the death of Lieutenant-General Stavrogin, a frivolous
old gentleman who died of a stomach ailment on the way to the Crimea,
where he was hastening to join the army on active service. Varvara
Petrovna was left a widow and put on deep mourning. She could not, it is
true, deplore his death very deeply, since, for the last four years,
she had been completely separated from him owing to incompatibility of
temper, and was giving him an allowance. (The Lieutenant-General himself
had nothing but one hundred and fifty serfs and his pay, besides his
position and his connections. All the money and Skvoreshniki belonged to
Varvara Petrovna, the only daughter of a very rich contractor.) Yet she
was shocked by the suddenness of the news, and retired into complete
solitude. Stepan Trofimovitch, of course, was always at her side.


\vspace{12pt}
May was in its full beauty. The evenings were exquisite. The wild cherry
was in flower. The two friends walked every evening in the garden and
used to sit till nightfall in the arbour, and pour out their thoughts
and feelings to one another. They had poetic moments. Under the
influence of the change in her position Varvara Petrovna talked more
than usual. She, as it were, clung to the heart of her friend, and this
continued for several evenings. A strange idea suddenly came over Stepan
Trofimovitch: ``Was not the inconsolable widow reckoning upon him, and
expecting from him, when her mourning was over, the offer of his hand?"
A cynical idea, but the very loftiness of a man's nature sometimes
increases a disposition to cynical ideas if only from the many-sidedness
of his culture. He began to look more deeply into it, and thought it
seemed like it. He pondered: ``Her fortune is immense, of course, but..."
Varvara Petrovna certainly could not be called a beauty. She was a
tall, yellow, bony woman with an extremely long face, suggestive of a
horse. Stepan Trofimovitch hesitated more and more, he was tortured by
doubts, he positively shed tears of indecision once or twice (he wept
not infrequently). In the evenings, that is to say in the arbour, his
countenance involuntarily began to express something capricious and
ironical, something coquettish and at the same time condescending. This
is apt to happen as it were by accident, and the more gentlemanly the
man the more noticeable it is. Goodness only knows what one is to think
about it, but it's most likely that nothing had begun working in her
heart that could have fully justified Stepan Trofimovitch's suspicions.
Moreover, she would not have changed her name, Stavrogin, for his
name, famous as it was. Perhaps there was nothing in it but the play
of femininity on her side; the manifestation of an unconscious feminine
yearning so natural in some extremely feminine types. However, I won't
answer for it; the depths of the female heart have not been explored to
this day. But I must continue.


\vspace{12pt}
It is to be supposed that she soon inwardly guessed the significance of
her friend's strange expression; she was quick and observant, and he was
sometimes extremely guileless. But the evenings went on as before, and
their conversations were just as poetic and interesting. And behold
on one occasion at nightfall, after the most lively and poetical
conversation, they parted affectionately, warmly pressing each other's
hands at the steps of the lodge where Stepan Trofimovitch slept. Every
summer he used to move into this little lodge which stood adjoining the
huge seignorial house of Skvoreshniki, almost in the garden. He had only
just gone in, and in restless hesitation taken a cigar, and not having
yet lighted it, was standing weary and motionless before the open
window, gazing at the light feathery white clouds gliding around the
bright moon, when suddenly a faint rustle made him start and turn
round. Varvara Petrovna, whom he had left only four minutes earlier,
was standing before him again. Her yellow face was almost blue. Her lips
were pressed tightly together and twitching at the corners. For ten full
seconds she looked him in the eyes in silence with a firm relentless
gaze, and suddenly whispered rapidly:


\vspace{12pt}
``I shall never forgive you for this!"


\vspace{12pt}
When, ten years later, Stepan Trofimovitch, after closing the doors,
told me this melancholy tale in a whisper, he vowed that he had been so
petrified on the spot that he had not seen or heard how Varvara Petrovna
had disappeared. As she never once afterwards alluded to the incident
and everything went on as though nothing had happened, he was all his
life inclined to the idea that it was all an hallucination, a symptom
of illness, the more so as he was actually taken ill that very night
and was indisposed for a fortnight, which, by the way, cut short the
interviews in the arbour.


\vspace{12pt}
But in spite of his vague theory of hallucination he seemed every day,
all his life, to be expecting the continuation, and, so to say, the
\emph{dénouement} of this affair. He could not believe that that was the end of
it! And if so he must have looked strangely sometimes at his friend.



\vspace{12pt}
V


\vspace{12pt}
She had herself designed the costume for him which he wore for the rest
of his life. It was elegant and characteristic; a long black frock-coat,
buttoned almost to the top, but stylishly cut; a soft hat (in summer a
straw hat) with a wide brim, a white batiste cravat with a full bow
and hanging ends, a cane with a silver knob; his hair flowed on to his
shoulders. It was dark brown, and only lately had begun to get a little
grey. He was clean-shaven. He was said to have been very handsome in his
youth. And, to my mind, he was still an exceptionally impressive figure
even in old age. Besides, who can talk of old age at fifty-three?
From his special pose as a patriot, however, he did not try to appear
younger, but seemed rather to pride himself on the solidity of his
age, and, dressed as described, tall and thin with flowing hair, he
looked almost like a patriarch, or even more like the portrait of the
poet Kukolnik, engraved in the edition of his works published in 1830 or
thereabouts. This resemblance was especially striking when he sat in the
garden in summertime, on a seat under a bush of flowering lilac, with
both hands propped on his cane and an open book beside him, musing
poetically over the setting sun. In regard to books I may remark that
he came in later years rather to avoid reading. But that was only quite
towards the end. The papers and magazines ordered in great profusion by
Varvara Petrovna he was continually reading. He never lost interest in
the successes of Russian literature either, though he always maintained
a dignified attitude with regard to them. He was at one time engrossed
in the study of our home and foreign politics, but he soon gave up the
undertaking with a gesture of despair. It sometimes happened that he
would take De Tocqueville with him into the garden while he had a Paul
de Kock in his pocket. But these are trivial matters.


\vspace{12pt}
I must observe in parenthesis about the portrait of Kukolnik; the
engraving had first come into the hands of Varvara Petrovna when she was
a girl in a high-class boarding-school in Moscow. She fell in love with
the portrait at once, after the habit of all girls at school who fall
in love with anything they come across, as well as with their teachers,
especially the drawing and writing masters. What is interesting in this,
though, is not the characteristics of girls but the fact that even at
fifty Varvara Petrovna kept the engraving among her most intimate and
treasured possessions, so that perhaps it was only on this account that
she had designed for Stepan Trofimovitch a costume somewhat like the
poet's in the engraving. But that, of course, is a trifling matter too.


\vspace{12pt}
For the first years or, more accurately, for the first half of the time
he spent with Varvara Petrovna, Stepan Trofimovitch was still planning a
\begin{figure}[!ht]
\begin{center}
\includegraphics[height=4.1in]{demons-5.jpg}
\end{center}
\end{figure}
book and every day seriously prepared to write it. But during the later
period he must have forgotten even what he had done. More and more
frequently he used to say to us:


\vspace{12pt}
``I seem to be ready for work, my materials are collected, yet the work
doesn't get done! Nothing is done!"


\vspace{12pt}
And he would bow his head dejectedly. No doubt this was calculated
to increase his prestige in our eyes as a martyr to science, but he
himself was longing for something else. ``They have forgotten me! I'm
no use to anyone!" broke from him more than once. This intensified
depression took special hold of him towards the end of the fifties.
Varvara Petrovna realised at last that it was a serious matter. Besides,
she could not endure the idea that her friend was forgotten and useless.
To distract him and at the same time to renew his fame she carried him
off to Moscow, where she had fashionable acquaintances in the
literary and scientific world; but it appeared that Moscow too was
unsatisfactory.


\vspace{12pt}
It was a peculiar time; something new was beginning, quite unlike the
stagnation of the past, something very strange too, though it was felt
everywhere, even at Skvoreshniki. Rumours of all sorts reached us. The
facts were generally more or less well known, but it was evident that
in addition to the facts there were certain ideas accompanying them,
and what's more, a great number of them. And this was perplexing. It was
impossible to estimate and find out exactly what was the drift of these
ideas. Varvara Petrovna was prompted by the feminine composition of her
character to a compelling desire to penetrate the secret of them.
She took to reading newspapers and magazines, prohibited publications
printed abroad and even the revolutionary manifestoes which were just
beginning to appear at the time (she was able to procure them all); but
this only set her head in a whirl. She fell to writing letters; she got
few answers, and they grew more incomprehensible as time went on. Stepan
Trofimovitch was solemnly called upon to explain `these ideas' to
her once for all, but she remained distinctly dissatisfied with his
explanations.


\vspace{12pt}
Stepan Trofimovitch's view of the general movement was supercilious in
the extreme. In his eyes all it amounted to was that he was forgotten
and of no use. At last his name was mentioned, at first in periodicals
published abroad as that of an exiled martyr, and immediately afterwards
in Petersburg as that of a former star in a celebrated constellation.
He was even for some reason compared with Radishtchev. Then some one
printed the statement that he was dead and promised an obituary notice
of him. Stepan Trofimovitch instantly perked up and assumed an air of
immense dignity. All his disdain for his contemporaries evaporated and
he began to cherish the dream of joining the movement and showing his
powers. Varvara Petrovna's faith in everything instantly revived and she
was thrown into a violent ferment. It was decided to go to Petersburg
without a moment's delay, to find out everything on the spot, to go into
everything personally, and, if possible, to throw themselves heart and
soul into the new movement. Among other things she announced that she
was prepared to found a magazine of her own, and henceforward to devote
her whole life to it. Seeing what it had come to, Stepan Trofimovitch
became more condescending than ever, and on the journey began to behave
almost patronisingly to Varvara Petrovna--which she at once laid up in
her heart against him. She had, however, another very important reason
for the trip, which was to renew her connections in higher spheres.
It was necessary, as far as she could, to remind the world of her
existence, or at any rate to make an attempt to do so. The ostensible
object of the journey was to see her only son, who was just finishing
his studies at a Petersburg lyceum.



\vspace{12pt}
VI


\vspace{12pt}
They spent almost the whole winter season in Petersburg. But by Lent
everything burst like a rainbow-coloured soap-bubble.


\vspace{12pt}
Their dreams were dissipated, and the muddle, far from being cleared
up, had become even more revoltingly incomprehensible. To begin with,
connections with the higher spheres were not established, or only on a
microscopic scale, and by humiliating exertions. In her mortification
Varvara Petrovna threw herself heart and soul into the `new ideas,' and
began giving evening receptions. She invited literary people, and they
were brought to her at once in multitudes. Afterwards they came of
themselves without invitation, one brought another. Never had she seen
such literary men. They were incredibly vain, but quite open in their
vanity, as though they were performing a duty by the display of it.
Some (but by no means all) of them even turned up intoxicated, seeming,
however, to detect in this a peculiar, only recently discovered, merit.
They were all strangely proud of something. On every face was written
that they had only just discovered some extremely important secret. They
abused one another, and took credit to themselves for it. It was rather
difficult to find out what they had written exactly, but among them
there were critics, novelists, dramatists, satirists, and exposers of
abuses. Stepan Trofimovitch penetrated into their very highest circle
from which the movement was directed. Incredible heights had to be
scaled to reach this group; but they gave him a cordial welcome, though,
of course, no one of them had ever heard of him or knew anything about
him except that he ``represented an idea." His manœuvres among them
were so successful that he got them twice to Varvara Petrovna's salon
in spite of their Olympian grandeur. These people were very serious and
very polite; they behaved nicely; the others were evidently afraid of
them; but it was obvious that they had no time to spare. Two or three
former literary celebrities who happened to be in Petersburg, and with
whom Varvara Petrovna had long maintained a most refined correspondence,
came also. But to her surprise these genuine and quite indubitable
celebrities were stiller than water, humbler than the grass, and some
of them simply hung on to this new rabble, and were shamefully cringing
before them. At first Stepan Trofimovitch was a success. People caught
at him and began to exhibit him at public literary gatherings. The first
time he came on to the platform at some public reading in which he was
to take part, he was received with enthusiastic clapping which lasted
for five minutes. He recalled this with tears nine years afterwards,
though rather from his natural artistic sensibility than from gratitude.
``I swear, and I'm ready to bet," he declared (but only to me, and in
secret), ``that not one of that audience knew anything whatever about
me." A noteworthy admission. He must have had a keen intelligence since
he was capable of grasping his position so clearly even on the platform,
even in such a state of exaltation; it also follows that he had not
a keen intelligence if, nine years afterwards, he could not recall
it without mortification, he was made to sign two or three collective
protests (against what he did not know); he signed them. Varvara
Petrovna too was made to protest against some `disgraceful action' and
she signed too. The majority of these new people, however, though they
visited Varvara Petrovna, felt themselves for some reason called upon
to regard her with contempt, and with undisguised irony. Stepan
Trofimovitch hinted to me at bitter moments afterwards that it was from
that time she had been envious of him. She saw, of course, that she
could not get on with these people, yet she received them eagerly,
with all the hysterical impatience of her sex, and, what is more, she
expected something. At her parties she talked little, although she could
talk, but she listened the more. They talked of the abolition of the
censorship, and of phonetic spelling, of the substitution of the Latin
characters for the Russian alphabet, of some one's having been sent into
exile the day before, of some scandal, of the advantage of splitting
Russia into nationalities united in a free federation, of the abolition
of the army and the navy, of the restoration of Poland as far as
the Dnieper, of the peasant reforms, and of the manifestoes, of the
abolition of the hereditary principle, of the family, of children, and
\begin{figure}[!ht]
\begin{center}
\includegraphics[height=4.1in]{demons-6.jpg}
\end{center}
\end{figure}
of priests, of women's rights, of Kraevsky's house, for which no one
ever seemed able to forgive Mr. Kraevsky, and so on, and so on. It was
evident that in this mob of new people there were many impostors, but
undoubtedly there were also many honest and very attractive people, in
spite of some surprising characteristics in them. The honest ones were
far more difficult to understand than the coarse and dishonest, but it
was impossible to tell which was being made a tool of by the other.
When Varvara Petrovna announced her idea of founding a magazine, people
flocked to her in even larger numbers, but charges of being a capitalist
and an exploiter of labour were showered upon her to her face. The
rudeness of these accusations was only equalled by their unexpectedness.
The aged General Ivan Ivanovitch Drozdov, an old friend and comrade
of the late General Stavrogin's, known to us all here as an extremely
stubborn and irritable, though very estimable, man (in his own way, of
course), who ate a great deal, and was dreadfully afraid of atheism,
quarrelled at one of Varvara Petrovna's parties with a distinguished
young man. The latter at the first word exclaimed, ``You must be a
general if you talk like that," meaning that he could find no word of
abuse worse than ``general."


\vspace{12pt}
Ivan Ivanovitch flew into a terrible passion: ``Yes, sir, I am a general,
and a lieutenant-general, and I have served my Tsar, and you, sir, are a
puppy and an infidel!"


\vspace{12pt}
An outrageous scene followed. Next day the incident was exposed in
print, and they began getting up a collective protest against Varvara
Petrovna's disgraceful conduct in not having immediately turned
the general out. In an illustrated paper there appeared a malignant
caricature in which Varvara Petrovna, Stepan Trofimovitch, and General
Drozdov were depicted as three reactionary friends. There were verses
attached to this caricature written by a popular poet especially for the
occasion. I may observe, for my own part, that many persons of general's
rank certainly have an absurd habit of saying, ``I have served my
Tsar"...just as though they had not the same Tsar as all the rest of us,
their simple fellow-subjects, but had a special Tsar of their own.


\vspace{12pt}
It was impossible, of course, to remain any longer in Petersburg, all
the more so as Stepan Trofimovitch was overtaken by a complete fiasco.
He could not resist talking of the claims of art, and they laughed
at him more loudly as time went on. At his last lecture he thought to
impress them with patriotic eloquence, hoping to touch their hearts,
and reckoning on the respect inspired by his ``persecution." He did
not attempt to dispute the uselessness and absurdity of the word
`fatherland,' acknowledged the pernicious influence of religion, but
firmly and loudly declared that boots were of less consequence than
Pushkin; of much less, indeed. He was hissed so mercilessly that he
burst into tears, there and then, on the platform. Varvara Petrovna took
him home more dead than alive. \emph{``On m'a traité comme un vieux bonnet
de coton,"} he babbled senselessly. She was looking after him all night,
giving him laurel-drops and repeating to him till daybreak, ``You will
still be of use; you will still make your mark; you will be appreciated
... in another place."


\vspace{12pt}
Early next morning five literary men called on Varvara Petrovna, three
of them complete strangers, whom she had never set eyes on before. With
a stern air they informed her that they had looked into the question of
her magazine, and had brought her their decision on the subject. Varvara
Petrovna had never authorised anyone to look into or decide anything
concerning her magazine. Their decision was that, having founded the
magazine, she should at once hand it over to them with the capital to
run it, on the basis of a co-operative society. She herself was to
go back to Skvoreshniki, not forgetting to take with her Stepan
Trofimovitch, who was ``out of date." From delicacy they agreed to
recognise the right of property in her case, and to send her every year
a sixth part of the net profits. What was most touching about it
was that of these five men, four certainly were not actuated by any
mercenary motive, and were simply acting in the interests of the
``cause."


\vspace{12pt}
`We came away utterly at a loss,' Stepan Trofimovitch used to say
afterwards. ``I couldn't make head or tail of it, and kept muttering, I
remember, to the rumble of the train:


\vspace{12pt}
     'Vyek, and vyek, and Lyov Kambek,
     Lyov Kambek and vyek, and vyek.'


\vspace{12pt}
and goodness knows what, all the way to Moscow. It was only in Moscow
that I came to myself--as though we really might find something
different there."


\vspace{12pt}
``Oh, my friends!` he would exclaim to us sometimes with fervour, 'you
cannot imagine what wrath and sadness overcome your whole soul when a
great idea, which you have long cherished as holy, is caught up by the
ignorant and dragged forth before fools like themselves into the street,
and you suddenly meet it in the market unrecognisable, in the mud,
absurdly set up, without proportion, without harmony, the plaything of
foolish louts! No! In our day it was not so, and it was not this for
which we strove. No, no, not this at all. I don't recognise it.... Our
day will come again and will turn all the tottering fabric of to-day
into a true path. If not, what will happen?..."



\vspace{12pt}
VII


\vspace{12pt}
Immediately on their return from Petersburg Varvara Petrovna sent her
friend abroad to `recruit'; and, indeed, it was necessary for them to
part for a time, she felt that. Stepan Trofimovitch was delighted to go.


\vspace{12pt}
``There I shall revive!" he exclaimed. ``There, at last, I shall set to
work!" But in the first of his letters from Berlin he struck his usual
note:


\vspace{12pt}
``My heart is broken!" he wrote to Varvara Petrovna. ``I can forget
nothing! Here, in Berlin, everything brings back to me my old past, my
first raptures and my first agonies. Where is she? Where are they both?
Where are you two angels of whom I was never worthy? Where is my son, my
beloved son? And last of all, where am I, where is my old self, strong
as steel, firm as a rock, when now some Andreev, our orthodox clown with
a beard, \emph{peut briser mon existence en deux}"--and so on.


\vspace{12pt}
As for Stepan Trofimovitch's son, he had only seen him twice in his
life, the first time when he was born and the second time lately in
Petersburg, where the young man was preparing to enter the university.
The boy had been all his life, as we have said already, brought up by
his aunts (at Varvara Petrovna's expense) in a remote province, nearly
six hundred miles from Skvoreshniki. As for Andreev, he was nothing
more or less than our local shopkeeper, a very eccentric fellow, a
self-taught archæologist who had a passion for collecting Russian
antiquities and sometimes tried to outshine Stepan Trofimovitch in
\begin{figure}[!ht]
\begin{center}
\includegraphics[height=4.1in]{demons-7.jpg}
\end{center}
\end{figure}
erudition and in the progressiveness of his opinions. This worthy
shopkeeper, with a grey beard and silver-rimmed spectacles, still owed
Stepan Trofimovitch four hundred roubles for some acres of timber he had
bought on the latter's little estate (near Skvoreshniki). Though Varvara
Petrovna had liberally provided her friend with funds when she sent him
to Berlin, yet Stepan Trofimovitch had, before starting, particularly
reckoned on getting that four hundred roubles, probably for his secret
expenditure, and was ready to cry when Andreev asked leave to defer
payment for a month, which he had a right to do, since he had brought
the first installments of the money almost six months in advance to meet
Stepan Trofimovitch's special need at the time.


\vspace{12pt}
Varvara Petrovna read this first letter greedily, and underlining in
pencil the exclamation: ``Where are they both?" numbered it and put it
away in a drawer. He had, of course, referred to his two deceased wives.
The second letter she received from Berlin was in a different strain:


\vspace{12pt}
``I am working twelve hours out of the twenty-four." (``Eleven would be
enough," muttered Varvara Petrovna.) ``I'm rummaging in the libraries,
collating, copying, rushing about. I've visited the professors. I have
renewed my acquaintance with the delightful Dundasov family. What a
charming creature Lizaveta Nikolaevna is even now! She sends you her
greetings. Her young husband and three nephews are all in Berlin. I
sit up talking till daybreak with the young people and we have almost
Athenian evenings, Athenian, I mean, only in their intellectual subtlety
and refinement. Everything is in noble style; a great deal of music,
Spanish airs, dreams of the regeneration of all humanity, ideas
of eternal beauty, of the Sistine Madonna, light interspersed with
darkness, but there are spots even on the sun! Oh, my friend, my noble,
faithful friend! In heart I am with you and am yours; with you alone,
always, \emph{en tout pays}, even in \emph{le pays de Makar et de ses veaux}, of
which we often used to talk in agitation in Petersburg, do you remember,
before we came away. I think of it with a smile. Crossing the frontier I
felt myself in safety, a sensation, strange and new, for the first time
after so many years"--and so on and so on.


\vspace{12pt}
``Come, it's all nonsense!" Varvara Petrovna commented, folding up that
letter too. ``If he's up till daybreak with his Athenian nights, he isn't
at his books for twelve hours a day. Was he drunk when he wrote it?
That Dundasov woman dares to send me greetings! But there, let him amuse
himself!"


\vspace{12pt}
The phrase "\emph{dans le pays de Makar et de ses veaux}" meant: ``wherever
Makar may drive his calves." Stepan Trofimovitch sometimes purposely
translated Russian proverbs and traditional sayings into French in the
most stupid way, though no doubt he was able to understand and translate
them better. But he did it from a feeling that it was chic, and thought
it witty.


\vspace{12pt}
But he did not amuse himself for long. He could not hold out for four
months, and was soon flying back to Skvoreshniki. His last letters
consisted of nothing but outpourings of the most sentimental love for
his absent friend, and were literally wet with tears. There are natures
extremely attached to home like lap-dogs. The meeting of the friends was
enthusiastic. Within two days everything was as before and even duller
than before. `My friend,' Stepan Trofimovitch said to me a fortnight
after, in dead secret, ``I have discovered something awful for me...
something new: \emph{je suis un simple} dependent, \emph{et rien de plus! Mais
r-r-rien de plus.}"



\vspace{12pt}
VIII


\vspace{12pt}
After this we had a period of stagnation which lasted nine years.
The hysterical outbreaks and sobbings on my shoulder that recurred at
regular intervals did not in the least mar our prosperity. I wonder that
Stepan Trofimovitch did not grow stout during this period. His nose was
a little redder, and his manner had gained in urbanity, that was all. By
degrees a circle of friends had formed around him, although it was never
a very large one. Though Varvara Petrovna had little to do with the
circle, yet we all recognised her as our patroness. After the lesson she
had received in Petersburg, she settled down in our town for good. In
winter she lived in her town house and spent the summer on her estate
in the neighbourhood. She had never enjoyed so much consequence and
prestige in our provincial society as during the last seven years of
this period, that is up to the time of the appointment of our present
governor. Our former governor, the mild Ivan Ossipovitch, who will never
be forgotten among us, was a near relation of Varvara Petrovna's, and
had at one time been under obligations to her. His wife trembled at the
very thought of displeasing her, while the homage paid her by provincial
society was carried almost to a pitch that suggested idolatry. So Stepan
Trofimovitch, too, had a good time. He was a member of the club, lost at
cards majestically, and was everywhere treated with respect, though
many people regarded him only as a ``learned man." Later on, when Varvara
Petrovna allowed him to live in a separate house, we enjoyed greater
freedom than before. Twice a week we used to meet at his house. We were
a merry party, especially when he was not sparing of the champagne. The
wine came from the shop of the same Andreev. The bill was paid twice
a year by Varvara Petrovna, and on the day it was paid Stepan
Trofimovitch almost invariably suffered from an attack of his ``summer
cholera."


\vspace{12pt}
One of the first members of our circle was Liputin, an elderly
provincial official, and a great liberal, who was reputed in the town
to be an atheist. He had married for the second time a young and pretty
wife with a dowry, and had, besides, three grown-up daughters. He
brought up his family in the fear of God, and kept a tight hand over
them. He was extremely stingy, and out of his salary had bought himself
a house and amassed a fortune. He was an uncomfortable sort of man, and
had not been in the service. He was not much respected in the town, and
was not received in the best circles. Moreover, he was a scandal-monger,
and had more than once had to smart for his back-biting, for which he
had been badly punished by an officer, and again by a country gentleman,
the respectable head of a family. But we liked his wit, his inquiring
mind, his peculiar, malicious liveliness. Varvara Petrovna disliked him,
but he always knew how to make up to her.


\vspace{12pt}
Nor did she care for Shatov, who became one of our circle during the
last years of this period. Shatov had been a student and had been
expelled from the university after some disturbance. In his childhood he
had been a student of Stepan Trofimovitch's and was by birth a serf of
Varvara Petrovna's, the son of a former valet of hers, Pavel Fyodoritch,
and was greatly indebted to her bounty. She disliked him for his pride
and ingratitude and could never forgive him for not having come straight
to her on his expulsion from the university. On the contrary he had not
even answered the letter she had expressly sent him at the time, and
preferred to be a drudge in the family of a merchant of the new style,
with whom he went abroad, looking after his children more in the
position of a nurse than of a tutor. He was very eager to travel at the
time. The children had a governess too, a lively young Russian lady, who
also became one of the household on the eve of their departure, and
had been engaged chiefly because she was so cheap. Two months later the
merchant turned her out of the house for ``free thinking." Shatov took
himself off after her and soon afterwards married her in Geneva.
They lived together about three weeks, and then parted as free people
recognising no bonds, though, no doubt, also through poverty. He
wandered about Europe alone for a long time afterwards, living God knows
\begin{figure}[!ht]
\begin{center}
\includegraphics[height=4.1in]{demons-8.jpg}
\end{center}
\end{figure}
how; he is said to have blacked boots in the street, and to have been a
porter in some dockyard. At last, a year before, he had returned to his
native place among us and settled with an old aunt, whom he buried a
month later. His sister Dasha, who had also been brought up by Varvara
Petrovna, was a favourite of hers, and treated with respect and
consideration in her house. He saw his sister rarely and was not on
intimate terms with her. In our circle he was always sullen, and never
talkative; but from time to time, when his convictions were touched
upon, he became morbidly irritable and very unrestrained in his
language.


\vspace{12pt}
`One has to tie Shatov up and then argue with him,' Stepan Trofimovitch
would sometimes say in joke, but he liked him.


\vspace{12pt}
Shatov had radically changed some of his former socialistic convictions
abroad and had rushed to the opposite extreme. He was one of those
idealistic beings common in Russia, who are suddenly struck by some
overmastering idea which seems, as it were, to crush them at once, and
sometimes for ever. They are never equal to coping with it, but put
passionate faith in it, and their whole life passes afterwards, as it
were, in the last agonies under the weight of the stone that has fallen
upon them and half crushed them. In appearance Shatov was in complete
harmony with his convictions: he was short, awkward, had a shock of
flaxen hair, broad shoulders, thick lips, very thick overhanging white
eyebrows, a wrinkled forehead, and a hostile, obstinately downcast, as
it were shamefaced, expression in his eyes. His hair was always in a
wild tangle and stood up in a shock which nothing could smooth. He was
seven- or eight-and-twenty.


\vspace{12pt}
`I no longer wonder that his wife ran away from him,' Varvara Petrovna
enunciated on one occasion after gazing intently at him. He tried to be
neat in his dress, in spite of his extreme poverty. He refrained again
from appealing to Varvara Petrovna, and struggled along as best he
could, doing various jobs for tradespeople. At one time he served in a
shop, at another he was on the point of going as an assistant clerk on a
freight steamer, but he fell ill just at the time of sailing. It is
hard to imagine what poverty he was capable of enduring without thinking
about it at all. After his illness Varvara Petrovna sent him a hundred
roubles, anonymously and in secret. He found out the secret, however,
and after some reflection took the money and went to Varvara Petrovna to
thank her. She received him with warmth, but on this occasion, too,
he shamefully disappointed her. He only stayed five minutes, staring
blankly at the ground and smiling stupidly in profound silence, and
suddenly, at the most interesting point, without listening to what
she was saying, he got up, made an uncouth sideways bow, helpless
with confusion, caught against the lady's expensive inlaid work-table,
upsetting it on the floor and smashing it to atoms, and walked out
nearly dead with shame. Liputin blamed him severely afterwards for
having accepted the hundred roubles and having even gone to thank
Varvara Petrovna for them, instead of having returned the money with
contempt, because it had come from his former despotic mistress. He
lived in solitude on the outskirts of the town, and did not like any
of us to go and see him. He used to turn up invariably at Stepan
Trofimovitch's evenings, and borrowed newspapers and books from him.


\vspace{12pt}
There was another young man who always came, one Virginsky, a clerk in
the service here, who had something in common with Shatov, though on
the surface he seemed his complete opposite in every respect. He was a
`family man' too. He was a pathetic and very quiet young man though
he was thirty; he had considerable education though he was chiefly
self-taught. He was poor, married, and in the service, and supported the
aunt and sister of his wife. His wife and all the ladies of his family
professed the very latest convictions, but in rather a crude form.
It was a case of `an idea dragged forth into the street,' as Stepan
Trofimovitch had expressed it upon a former occasion. They got it
all out of books, and at the first hint coming from any of our little
progressive corners in Petersburg they were prepared to throw anything
overboard, so soon as they were advised to do so, Madame Virginsky
practised as a midwife in the town. She had lived a long while
in Petersburg as a girl. Virginsky himself was a man of rare
single-heartedness, and I have seldom met more honest fervour.


\vspace{12pt}
`I will never, never, abandon these bright hopes,' he used to say to me
with shining eyes. Of these `bright hopes' he always spoke quietly, in
a blissful half-whisper, as it were secretly. He was rather tall, but
extremely thin and narrow-shouldered, and had extraordinarily lank hair
of a reddish hue. All Stepan Trofimovitch's condescending gibes at
some of his opinions he accepted mildly, answered him sometimes very
seriously, and often nonplussed him. Stepan Trofimovitch treated him
very kindly, and indeed he behaved like a father to all of us. ``You are
all half-hearted chickens," he observed to Virginsky in joke. ``All
who are like you, though in you, Virginsky, I have not observed that
narrow-mindedness I found in Petersburg, \emph{chez ces séminaristes}. But
you're a half-hatched chicken all the same. Shatov would give anything
to hatch out, but he's half-hatched too."


\vspace{12pt}
``And I?" Liputin inquired.


\vspace{12pt}
``You're simply the golden mean which will get on anywhere in its own
way." Liputin was offended.


\vspace{12pt}
The story was told of Virginsky, and it was unhappily only too true,
that before his wife had spent a year in lawful wedlock with him she
announced that he was superseded and that she preferred Lebyadkin. This
Lebyadkin, a stranger to the town, turned out afterwards to be a very
dubious character, and not a retired captain as he represented himself
to be. He could do nothing but twist his moustache, drink, and chatter
the most inept nonsense that can possibly be imagined. This fellow, who
was utterly lacking in delicacy, at once settled in his house, glad to
live at another man's expense, ate and slept there and came, in the end,
to treating the master of the house with condescension. It was asserted
that when Virginsky's wife had announced to him that he was superseded
he said to her:


\vspace{12pt}
`My dear, hitherto I have only loved you, but now I respect you,' but I
doubt whether this renunciation, worthy of ancient Rome, was ever really
uttered. On the contrary they say that he wept violently. A fortnight
after he was superseded, all of them, in a `family party,' went one day
for a picnic to a wood outside the town to drink tea with their friends.
Virginsky was in a feverishly lively mood and took part in the dances.
But suddenly, without any preliminary quarrel, he seized the giant
Lebyadkin with both hands, by the hair, just as the latter was dancing
a can-can solo, pushed him down, and began dragging him along with
shrieks, shouts, and tears. The giant was so panic-stricken that he did
not attempt to defend himself, and hardly uttered a sound all the time
he was being dragged along. But afterwards he resented it with all the
heat of an honourable man. Virginsky spent a whole night on his knees
begging his wife's forgiveness. But this forgiveness was not granted, as
he refused to apologise to Lebyadkin; moreover, he was upbraided for the
meanness of his ideas and his foolishness, the latter charge based on
the fact that he knelt down in the interview with his wife. The captain
soon disappeared and did not reappear in our town till quite lately,
when he came with his sister, and with entirely different aims; but
of him later. It was no wonder that the poor young husband sought our
society and found comfort in it. But he never spoke of his home-life to
us. On one occasion only, returning with me from Stepan Trofimovitch's,
he made a remote allusion to his position, but clutching my hand at once
\begin{figure}[!ht]
\begin{center}
\includegraphics[height=4.1in]{demons-9.jpg}
\end{center}
\end{figure}
he cried ardently:


\vspace{12pt}
``It's of no consequence. It's only a personal incident. It's no
hindrance to the 'cause,' not the slightest!"


\vspace{12pt}
Stray guests visited our circle too; a Jew, called Lyamshin, and a
Captain Kartusov came. An old gentleman of inquiring mind used to come
at one time, but he died. Liputin brought an exiled Polish priest called
Slontsevsky, and for a time we received him on principle, but afterwards
we didn't keep it up.



\vspace{12pt}
IX


\vspace{12pt}
At one time it was reported about the town that our little circle was a
hotbed of nihilism, profligacy, and godlessness, and the rumour gained
more and more strength. And yet we did nothing but indulge in the most
harmless, agreeable, typically Russian, light-hearted liberal chatter.
`The higher liberalism' and the `higher liberal,' that is, a liberal
without any definite aim, is only possible in Russia.


\vspace{12pt}
Stepan Trofimovitch, like every witty man, needed a listener, and,
besides that, he needed the consciousness that he was fulfilling the
lofty duty of disseminating ideas. And finally he had to have some one
to drink champagne with, and over the wine to exchange light-hearted
views of a certain sort, about Russia and the `Russian spirit,' about
God in general, and the `Russian God' in particular, to repeat for the
hundredth time the same Russian scandalous stories that every one knew
and every one repeated. We had no distaste for the gossip of the town
which often, indeed, led us to the most severe and loftily moral
verdicts. We fell into generalising about humanity, made stern
reflections on the future of Europe and mankind in general,
authoritatively predicted that after Cæsarism France would at once sink
into the position of a second-rate power, and were firmly convinced that
this might terribly easily and quickly come to pass. We had long ago
predicted that the Pope would play the part of a simple archbishop in
a united Italy, and were firmly convinced that this thousand-year-old
question had, in our age of humanitarianism, industry, and railways,
become a trifling matter. But, of course, `Russian higher liberalism'
could not look at the question in any other way. Stepan Trofimovitch
sometimes talked of art, and very well, though rather abstractly. He
sometimes spoke of the friends of his youth--all names noteworthy in
the history of Russian progress. He talked of them with emotion and
reverence, though sometimes with envy. If we were very much bored, the
Jew, Lyamshin (a little post-office clerk), a wonderful performer on
the piano, sat down to play, and in the intervals would imitate a pig,
a thunderstorm, a confinement with the first cry of the baby, and so on,
and so on; it was only for this that he was invited, indeed. If we had
drunk a great deal--and that did happen sometimes, though not often--we
flew into raptures, and even on one occasion sang the `Marseillaise' in
chorus to the accompaniment of Lyamshin, though I don't know how it
went off. The great day, the nineteenth of February, we welcomed
enthusiastically, and for a long time beforehand drank toasts in its
honour. But that was long ago, before the advent of Shatov or Virginsky,
when Stepan Trofimovitch was still living in the same house with Varvara
Petrovna. For some time before the great day Stepan Trofimovitch
fell into the habit of muttering to himself well-known, though rather
far-fetched, lines which must have been written by some liberal
landowner of the past:


\vspace{12pt}
\emph{``The peasant with his axe is coming,}


\emph{Something terrible will happen."}




\vspace{12pt}
Something of that sort, I don't remember the exact words. Varvara
Petrovna overheard him on one occasion, and crying, ``Nonsense,
nonsense!" she went out of the room in a rage. Liputin, who happened to
be present, observed malignantly to Stepan Trofimovitch:


\vspace{12pt}
``It'll be a pity if their former serfs really do some mischief to
\emph{messieurs les} landowners to celebrate the occasion," and he drew his
forefinger round his throat.


\vspace{12pt}
``\emph{Cher ami,}` Stepan Trofimovitch observed, 'believe me that--this (he
repeated the gesture) will never be of any use to our landowners nor to
any of us in general. We shall never be capable of organising anything
even without our heads, though our heads hinder our understanding more
than anything."


\vspace{12pt}
I may observe that many people among us anticipated that something
extraordinary, such as Liputin predicted, would take place on the day
of the emancipation, and those who held this view were the so-called
`authorities' on the peasantry and the government. I believe Stepan
Trofimovitch shared this idea, so much so that almost on the eve of the
great day he began asking Varvara Petrovna's leave to go abroad; in fact
he began to be uneasy. But the great day passed, and some time
passed after it, and the condescending smile reappeared on Stepan
Trofimovitch's lips. In our presence he delivered himself of some
noteworthy thoughts on the character of the Russian in general, and the
Russian peasant in particular.


\vspace{12pt}
`Like hasty people we have been in too great a hurry with our peasants,'
he said in conclusion of a series of remarkable utterances. ``We have
made them the fashion, and a whole section of writers have for several
years treated them as though they were newly discovered curiosities. We
have put laurel-wreaths on lousy heads. The Russian village has given us
only 'Kamarinsky' in a thousand years. A remarkable Russian poet who was
also something of a wit, seeing the great Rachel on the stage for the
first time cried in ecstasy, 'I wouldn't exchange Rachel for a peasant!'
I am prepared to go further. I would give all the peasants in Russia
for one Rachel. It's high time to look things in the face more
soberly, and not to mix up our national rustic pitch with \emph{bouquet de
l'Impératrice.}"


\vspace{12pt}
Liputin agreed at once, but remarked that one had to perjure oneself and
praise the peasant all the same for the sake of being progressive, that
even ladies in good society shed tears reading `Poor Anton,' and that
some of them even wrote from Paris to their bailiffs that they were,
henceforward, to treat the peasants as humanely as possible.


\vspace{12pt}
It happened, and as ill-luck would have it just after the rumours of the
Anton Petrov affair had reached us, that there was some disturbance
in our province too, only about ten miles from Skvoreshniki, so that a
detachment of soldiers was sent down in a hurry.


\vspace{12pt}
This time Stepan Trofimovitch was so much upset that he even frightened
us. He cried out at the club that more troops were needed, that they
ought to be telegraphed for from another province; he rushed off to the
governor to protest that he had no hand in it, begged him not to allow
his name on account of old associations to be brought into it, and
offered to write about his protest to the proper quarter in Petersburg.
Fortunately it all passed over quickly and ended in nothing, but I was
surprised at Stepan Trofimovitch at the time.


\vspace{12pt}
Three years later, as every one knows, people were beginning to talk
of nationalism, and `public opinion' first came upon the scene. Stepan
Trofimovitch laughed a great deal.


\vspace{12pt}
`My friends,' he instructed us, ``if our nationalism has 'dawned' as
they keep repeating in the papers--it's still at school, at some German
'Peterschule,' sitting over a German book and repeating its everlasting
German lesson, and its German teacher will make it go down on its knees
when he thinks fit. I think highly of the German teacher. But nothing
has happened and nothing of the kind has dawned and everything is going
on in the old way, that is, as ordained by God. To my thinking that
should be enough for Russia, \emph{pour notre Sainte Russie}. Besides, all this
Slavism and nationalism is too old to be new. Nationalism, if you like,
has never existed among us except as a distraction for gentlemen's
clubs, and Moscow ones at that. I'm not talking of the days of Igor, of
course. And besides it all comes of idleness. Everything in Russia comes
of idleness, everything good and fine even. It all springs from the
charming, cultured, whimsical idleness of our gentry! I'm ready to
repeat it for thirty thousand years. We don't know how to live by our
own labour. And as for the fuss they're making now about the 'dawn'
of some sort of public opinion, has it so suddenly dropped from heaven
without any warning? How is it they don't understand that before we
can have an opinion of our own we must have work, our own work, our own
initiative in things, our own experience. Nothing is to be gained for
nothing. If we work we shall have an opinion of our own. But as we
never shall work, our opinions will be formed for us by those who have
hitherto done the work instead of us, that is, as always, Europe, the
everlasting Germans--our teachers for the last two centuries. Moreover,
Russia is too big a tangle for us to unravel alone without the Germans,
and without hard work. For the last twenty years I've been sounding the
alarm, and the summons to work. I've given up my life to that appeal,
and, in my folly I put faith in it. Now I have lost faith in it, but I
sound the alarm still, and shall sound it to the tomb. I will pull at
the bell-ropes until they toll for my own requiem!"


\vspace{12pt}
``Alas! We could do nothing but assent. We applauded our teacher and with
what warmth, indeed! And, after all, my friends, don't we still hear
to-day, every hour, at every step, the same `charming,' `clever,'
`liberal,' old Russian nonsense? Our teacher believed in God.


\vspace{12pt}
``I can't understand why they make me out an infidel here," he used to
say sometimes. ``I believe in God, \emph{mais distinguons}, I believe in Him as
a Being who is conscious of Himself in me only. I cannot believe as my
Nastasya (the servant) or like some country gentleman who believes 'to
be on the safe side,' or like our dear Shatov--but no, Shatov doesn't
come into it. Shatov believes 'on principle,' like a Moscow Slavophil.
As for Christianity, for all my genuine respect for it, I'm not a
Christian. I am more of an antique pagan, like the great Goethe, or
like an ancient Greek. The very fact that Christianity has failed to
understand woman is enough, as George Sand has so splendidly shown in
one of her great novels. As for the bowings, fasting and all the rest
of it, I don't understand what they have to do with me. However busy the
informers may be here, I don't care to become a Jesuit. In the year 1847
Byelinsky, who was abroad, sent his famous letter to Gogol, and warmly
reproached him for believing in some sort of God. \emph{Entre nous soit dit,} I
can imagine nothing more comic than the moment when Gogol (the Gogol of
that period!) read that phrase, and... the whole letter! But dismissing
the humorous aspect, and, as I am fundamentally in agreement, I point to
them and say--these were men! They knew how to love their people, they
knew how to suffer for them, they knew how to sacrifice everything for
them, yet they knew how to differ from them when they ought, and did not
filch certain ideas from them. Could Byelinsky have sought salvation
in Lenten oil, or peas with radish!..." But at this point Shatov
interposed.


\vspace{12pt}
``Those men of yours never loved the people, they didn't suffer for them,
and didn't sacrifice anything for them, though they may have amused
themselves by imagining it!" he growled sullenly, looking down, and
moving impatiently in his chair.


\vspace{12pt}
``They didn't love the people!" yelled Stepan Trofimovitch. ``Oh, how they
loved Russia!"


\vspace{12pt}
``Neither Russia nor the people!" Shatov yelled too, with flashing eyes.
``You can't love what you don't know and they had no conception of the
Russian people. All of them peered at the Russian people through their
fingers, and you do too; Byelinsky especially: from that very letter to
Gogol one can see it. Byelinsky, like the Inquisitive Man in Krylov's
fable, did not notice the elephant in the museum of curiosities, but
concentrated his whole attention on the French Socialist beetles; he did
not get beyond them. And yet perhaps he was cleverer than any of you.
You've not only overlooked the people, you've taken up an attitude of
disgusting contempt for them, if only because you could not imagine any
but the French people, the Parisians indeed, and were ashamed that the
Russians were not like them. That's the naked truth. And he who has
no people has no God. You may be sure that all who cease to understand
their own people and lose their connection with them at once lose to
the same extent the faith of their fathers, and become atheistic or
indifferent. I'm speaking the truth! This is a fact which will be
realised. That's why all of you and all of us now are either beastly
atheists or careless, dissolute imbeciles, and nothing more. And you
too, Stepan Trofimovitch, I don't make an exception of you at all! In
fact, it is on your account I am speaking, let me tell you that!"


\vspace{12pt}
As a rule, after uttering such monologues (which happened to him pretty
frequently) Shatov snatched up his cap and rushed to the door, in the
full conviction that everything was now over, and that he had cut short
all friendly relations with Stepan Trofimovitch for ever. But the latter
always succeeded in stopping him in time.


\vspace{12pt}
``Hadn't we better make it up, Shatov, after all these endearments," he
would say, benignly holding out his hand to him from his arm-chair.


\vspace{12pt}
Shatov, clumsy and bashful, disliked sentimentality. Externally he was
rough, but inwardly, I believe, he had great delicacy. Although he often
went too far, he was the first to suffer for it. Muttering something
between his teeth in response to Stepan Trofimovitch's appeal, and
shuffling with his feet like a bear, he gave a sudden and unexpected
smile, put down his cap, and sat down in the same chair as before, with
his eyes stubbornly fixed on the ground. Wine was, of course, brought
in, and Stepan Trofimovitch proposed some suitable toast, for instance
the memory of some leading man of the past.







\vfill
\begin{center}
{\fontfamily{phv}\selectfont 
Modern Hacker Library\\books.modernhacker.com}
\end{center}
\end{document}
