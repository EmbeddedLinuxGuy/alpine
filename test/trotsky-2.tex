
\documentclass[12pt]{article}
\title{Terrorism and Communism}
\usepackage[bmargin=.25in, lmargin=.25in, rmargin=.25in, tmargin=.25in, paperwidth=4.25in, paperheight=5.5in]{geometry}
\usepackage{graphicx}
\begin{document}
\thispagestyle{empty}
\begin{center}
\textbf{\Large Terrorism and Communism}\\
\textbf{Leon Trotsky}\\

\end{center}

\begin{center}
\textbf{\small II. \hspace{1em} THE DICTATORSHIP OF THE PROLETARIAT
}
\end{center}
\begin{center}
\begin{tiny}
\begin{verbatim}
@@@@@@@@@@@@@@@@@@@@@**^^""~~~"^@@^*@*@@**@@@@@@@@@
@@@@@@@@@@@@@*^^'"~   , - ' '; ,@@b. '  -e@@@@@@@@@
@@@@@@@@*^"~      . '     . ' ,@@@@(  e@*@@@@@@@@@@
@@@@@^~         .       .   ' @@@@@@, ~^@@@@@@@@@@@
@@@~ ,e**@@*e,  ,e**e, .    ' '@@@@@@e,  "*@@@@@'^@
@',e@@@@@@@@@@ e@@@@@@       ' '*@@@@@@    @@@'   0
@@@@@@@@@@@@@@@@@@@@@',e,     ;  ~^*^'    ;^~   ' 0
@@@@@@@@@@@@@@@^""^@@e@@@   .'           ,'   .'  @
@@@@@@@@@@@@@@'    '@@@@@ '         ,  ,e'  .    ;@
@@@@@@@@@@@@@' ,&&,  ^@*'     ,  .  i^"@e, ,e@e  @@
@@@@@@@@@@@@' ,@@@@,          ;  ,& !,,@@@e@@@@ e@@
@@@@@,~*@@*' ,@@@@@@e,   ',   e^~^@,   ~'@@@@@@,@@@
@@@@@@, ~" ,e@@@@@@@@@*e*@*  ,@e  @@""@e,,@@@@@@@@@
@@@@@@@@ee@@@@@@@@@@@@@@@" ,e@' ,e@' e@@@@@@@@@@@@@
@@@@@@@@@@@@@@@@@@@@@@@@" ,@" ,e@@e,,@@@@@@@@@@@@@@
@@@@@@@@@@@@@@@@@@@@@@@~ ,@@@,,0@@@@@@@@@@@@@@@@@@@
@@@@@@@@@@@@@@@@@@@@@@@@,,@@@@@@@@@@@@@@@@@@@@@@@@@
"""""""""""""""""""""""""""""""""""""""""""""""""""
\end{verbatim}
\end{tiny}
\end{center}
\pagebreak


\vspace{12pt}
``Marx and Engels hammered out the idea of the dictatorship of the
proletariat, which Engels stubbornly defended in 1891, shortly before
his death -- the idea that the political autocracy of the proletariat is
the sole form in which it can realize its control of the state."

\vspace{12pt}
That is what Kautsky wrote about ten years ago. The sole form of power
for the proletariat he considered to be not a Socialist majority in a
democratic parliament, but the political autocracy of the proletariat,
its dictatorship. And it is quite clear that, if our problem is the
abolition of private property in the means of production, the only
road to its solution lies through the concentration of State power in
its entirety in the hands of the proletariat, and the setting up for
the transitional period of an exceptional regime -- a regime in which
the ruling class is guided, not by general principles calculated for a
prolonged period, but by considerations of revolutionary policy.

\vspace{12pt}
The dictatorship is necessary because it is a case, not of partial
changes, but of the very existence of the bourgeoisie. No agreement is
possible on this ground. Only force can be the deciding factor. The
dictatorship of the proletariat does not exclude, of course, either
separate agreements, or considerable concessions, especially in
connection with the lower middle-class and the peasantry. But the
proletariat can only conclude these agreements after having gained
possession of the apparatus of power, and having guaranteed to itself
the possibility of independently deciding on which points to yield and
on which to stand firm, in the interests of the general Socialist
task.

\vspace{12pt}
Kautsky now repudiates the dictatorship of the proletariat at the very
outset, as the ``tyranny of the minority over the majority." That is,
he discerns in the revolutionary regime of the proletariat those very
features by which the honest Socialists of all countries invariably
describe the dictatorship of the exploiters, albeit masked by the
forms of democracy.

\vspace{12pt}
Abandoning the idea of a revolutionary dictatorship, Kautsky
transforms the question of the conquest of power by the proletariat
into a question of the conquest of a majority of votes by the
Social-Democratic Party in one of the electoral campaigns of the
future. Universal suffrage, according to the legal fiction of
parliamentarism, expresses the will of the citizens of all classes in
the nation, and, consequently, gives a possibility of attracting a
majority to the side of Socialism. While the theoretical possibility
has not been realized, the Socialist minority must submit to the
bourgeois majority. This fetishism of the parliamentary majority
represents a brutal repudiation, not only of the dictatorship of the
proletariat, but of Marxism and of the revolution altogether. If, in
principle, we are to subordinate Socialist policy to the parliamentary
mystery of majority and minority, it follows that, in countries where
formal democracy prevails, there is no place at all for the
revolutionary struggle. If the majority elected on the basis of
universal suffrage in Switzerland pass draconian legislation against
strikers, or if the executive elected by the will of a formal majority
in Northern America shoots workers, have the Swiss and American
workers the ``right" of protest by organizing a general strike?
Obviously, no. The political strike is a form of extra-parliamentary
pressure on the ``national will," as it has expressed itself through
universal suffrage. True, Kautsky himself, apparently, is ashamed to
go as far as the logic of his new position demands. Bound by some sort
of remnant of the past, he is obliged to acknowledge the possibility
of correcting universal suffrage by action. Parliamentary elections,
at all events in principle, never took the place, in the eyes of the
Social-Democrats, of the real class struggle, of its conflicts,
repulses, attacks, revolts; they were considered merely as a
contributory fact in this struggle, playing a greater part at one
period, a smaller at another, and no part at all in the period of
dictatorship.

\vspace{12pt}
In 1891, that is, not long before his death, Engels, as we just heard,
obstinately defended the dictatorship of the proletariat as the only
possible form of its control of the State. Kautsky himself more than
once repeated this definition. Hence, by the way, we can see what an
unworthy forgery is Kautsky's present attempt to throw back the
dictatorship of the proletariat at us as a purely Russian invention.

\vspace{12pt}
Who aims at the end cannot reject the means. The struggle must be
carried on with such intensity as actually to guarantee the supremacy
of the proletariat. If the Socialist revolution requires a
dictatorship -- ``the sole form in which the proletariat can achieve
control of the State" -- it follows that the dictatorship must be
guaranteed at all cost.

\vspace{12pt}
To write a pamphlet about dictatorship one needs an ink-pot and a pile
of paper, and possibly, in addition, a certain number of ideas in
one's head. But in order to establish and consolidate the
dictatorship, one has to prevent the bourgeoisie from undermining the
State power of the proletariat. Kautsky apparently thinks that this
can be achieved by tearful pamphlets. But his own experience ought to
have shown him that it is not sufficient to have lost all influence
with the proletariat, to acquire influence with the bourgeoisie.

\vspace{12pt}
It is only possible to safeguard the supremacy of the working class by
forcing the bourgeoisie accustomed to rule, to realize that it is too
dangerous an undertaking for it to revolt against the dictatorship of
the proletariat, to undermine it by conspiracies, sabotage,
insurrections, or the calling in of foreign troops. The bourgeoisie,
hurled from power, must be forced to obey. In what way? The priests
used to terrify the people with future penalties. We have no such
resources at our disposal. But even the priests' hell never stood
alone, but was always bracketed with the material fire of the Holy
Inquisition, and with the scorpions of the democratic State. Is it
possible that Kautsky is leaning to the idea that the bourgeoisie can
be held down with the help of the categorical imperative, which in his
last writings plays the part of the Holy Ghost? We, on our part, can
only promise him our material assistance if he decides to equip a
Kantian-humanitarian mission to the realms of Denikin and Kolchak. At
all events, there he would have the possibility of convincing himself
that the counter-revolutionaries are not naturally devoid of
character, and that, thanks to their six years' existence in the fire
and smoke of war, their character has managed to become thoroughly
hardened. Every White Guard has long ago acquired the simple truth
that it is easier to hang a Communist to the branch of a tree than to
convert him with a book of Kautsky's. These gentlemen have no
superstitious fear, either of the principles of democracy or of the
flames of hell -- the more so because the priests of the church and of
official learning act in collusion with them, and pour their combined
thunders exclusively on the heads of the Bolsheviks. The Russian White
Guards resemble the German and all other White Guards in this
respect -- that they cannot be convinced or shamed, but only terrorized
or crushed.

\vspace{12pt}
The man who repudiates terrorism in principle -- \emph{i.e.}, repudiates
measures of suppression and intimidation towards determined and armed
counter-revolution, must reject all idea of the political supremacy of
the working class and its revolutionary dictatorship. The man who
repudiates the dictatorship of the proletariat repudiates the
Socialist revolution, and digs the grave of Socialism.

\vspace{12pt}
                 *       *       *       *       *

\vspace{12pt}
At the present time, Kautsky has no theory of the social revolution.
Every time he tries to generalize his slanders against the revolution
and the dictatorship of the proletariat, he produces merely a
r�chauff� of the prejudices of Jaur�sism and Bernsteinism.

\vspace{12pt}
``The revolution of 1789," writes Kautsky, ``itself put an end to the
most important causes which gave it its harsh and violent character,
and prepared the way for milder forms of the future revolution." (Page
140.)[2] Let us admit this, though to do so we have to forget the June
days of 1848 and the horrors of the suppression of the Commune. Let us
admit that the great revolution of the eighteenth century, which by
measures of merciless terror destroyed the rule of absolutism, of
feudalism, and of clericalism, really prepared the way for more
peaceful and milder solutions of social problems. But, even if we
admit this purely liberal standpoint, even here our accuser will prove
to be completely in the wrong; for the Russian Revolution, which
culminated in the dictatorship of the proletariat, began with just
that work which was done in France at the end of the eighteenth
century. Our forefathers, in centuries gone by, did not take the
trouble to prepare the democratic way -- by means of revolutionary
terrorism -- for milder manners in our revolution. The ethical mandarin,
Kautsky, ought to take these circumstances into account, and accuse
our forefathers, not us.

\vspace{12pt}
          [2] Translator's Note -- For convenience sake, the references
          throughout have been altered to fall in the English
          translation of Kautsky's book. Mr. Kerridge's translation,
          however, has not been adhered to.

\vspace{12pt}
Kautsky, however, seems to make a little concession in this direction.
``True," he says, ``no man of insight could doubt that a military
monarchy like the German, the Austrian, or the Russian could be
overthrown only by violent methods. But in this connection there was
always less thought" (amongst whom?), ``of the bloody use of arms, and
more of the working class weapon peculiar to the proletariat -- the
mass strike. And that a considerable portion of the proletariat,
after seizing power, would again -- as at the end of the eighteenth
century -- give vent to its rage and revenge in bloodshed could not be
expected. This would have meant a complete negation of all progress."
(Page 147.)

\vspace{12pt}
As we see, the war and a series of revolutions were required to enable
us to get a proper view of what was going on in reality in the heads of
some of our most learned theoreticians. It turns out that Kautsky did
not think that a Romanoff or a Hohenzollern could be put away by means
of conversations; but at the same time he seriously imagined that a
military monarchy could be overthrown by a general strike -- \emph{i.e.}, by
a peaceful demonstration of folded arms. In spite of the Russian
revolution, and the world discussion of this question, Kautsky, it
turns out, retains the anarcho-reformist view of the general strike.
We might point out to him that, in the pages of its own journal, the
\emph{Neue Zeit}, it was explained twelve years ago that the general strike
is only a mobilization of the proletariat and its setting up against
its enemy, the State; but that the strike in itself cannot produce
the solution of the problem, because it exhausts the forces of the
proletariat sooner than those of its enemies, and this, sooner or
later, forces the workers to return to the factories. The general
strike acquires a decisive importance only as a preliminary to a
conflict between the proletariat and the armed forces of the
opposition -- \emph{i.e.}, to the open revolutionary rising of the workers.
Only by breaking the will of the armies thrown against it can the
revolutionary class solve the problem of power -- the root problem of
every revolution. The general strike produces the mobilization of both
sides, and gives the first serious estimate of the powers of resistance
of the counter-revolution. But only in the further stages of the
struggle, after the transition to the path of armed insurrection, can
that bloody price be fixed which the revolutionary class has to pay for
power. But that it will have to pay with blood, that, in the struggle
for the conquest of power and for its consolidation, the proletariat
will have not only to be killed, but also to kill -- of this no serious
revolutionary ever had any doubt. To announce that the existence of a
determined life-and-death struggle between the proletariat and the
bourgeoisie ``is a complete negation of all progress," means simply that
the heads of some of our most reverend theoreticians take the form of a
camera-obscura, in which objects are represented upside down.

\vspace{12pt}
But, even when applied to more advanced and cultured countries with
established democratic traditions, there is absolutely no proof of
the justice of Kautsky's historical argument. As a matter of fact, the
argument itself is not new. Once upon a time the Revisionists gave it a
character more based on principle. They strove to prove that the growth
of proletarian organizations under democratic conditions guaranteed the
gradual and imperceptible -- reformist and evolutionary -- transition to
Socialist society -- without general strikes and risings, without the
dictatorship of the proletariat.

\vspace{12pt}
Kautsky, at that culminating period of his activity, showed that,
in spite of the forms of democracy, the class contradictions of
capitalist society grew deeper, and that this process must inevitably
lead to a revolution and the conquest of power by the proletariat.

\vspace{12pt}
No one, of course, attempted to reckon up beforehand the number of
victims that will be called for by the revolutionary insurrection of
the proletariat, and by the regime of its dictatorship. But it was
clear to all that the number of victims will vary with the strength of
resistance of the propertied classes. If Kautsky desires to say in his
book that a democratic upbringing has not weakened the class egoism of
the bourgeoisie, this can be admitted without further parley.

\vspace{12pt}
If he wishes to add that the imperialist war, which broke out and
continued for four years, \emph{in spite of} democracy, brought about
a degradation of morals and accustomed men to violent methods and
action, and completely stripped the bourgeoisie of the last vestige of
awkwardness in ordering the destruction of masses of humanity -- here
also he will be right.

\vspace{12pt}
All this is true on the face of it. But one has to struggle in real
conditions. The contending forces are not proletarian and bourgeois
manikins produced in the retort of Wagner-Kautsky, but a real
proletariat against a real bourgeoisie, as they have emerged from the
last imperialist slaughter.

\vspace{12pt}
In this fact of merciless civil war that is spreading over the whole
world, Kautsky sees only the result of a fatal lapse from the
``experienced tactics" of the Second International.

\vspace{12pt}
``In reality, since the time," he writes, ``that Marxism has dominated
the Socialist movement, the latter, up to the world war, was, in spite
of its great activities, preserved from great defeats. And the idea of
insuring victory by means of terrorist domination had completely
disappeared from its ranks.

\vspace{12pt}
``Much was contributed in this connection by the fact that, at the time
when Marxism was the dominating Socialist teaching, democracy threw
out firm roots in Western Europe, and began there to change from an
end of the struggle to a trustworthy basis of political life." (Page
145.)

\vspace{12pt}
In this ``formula of progress" there is not one atom of Marxism. The
real process of the struggle of classes and their material conflicts
has been lost in Marxist propaganda, which, thanks to the conditions
of democracy, guarantees, forsooth, a painless transition to a new and
``wiser" order. This is the most vulgar liberalism, a belated piece of
rationalism in the spirit of the eighteenth century -- with the
difference that the ideas of Condorcet are replaced by a vulgarisation
of the Communist Manifesto. All history resolves itself into an
endless sheet of printed paper, and the centre of this ``humane"
process proves to be the well-worn writing table of Kautsky.

\vspace{12pt}
We are given as an example the working-class movement in the period of
the Second International, which, going forward under the banner of
Marxism, never sustained great defeats whenever it deliberately
challenged them. But did not the whole working-class movement, the
proletariat of the whole world, and with it the whole of human
culture, sustain an incalculable defeat in August, 1914, when history
cast up the accounts of all the forces and possibilities of the
Socialist parties, amongst whom, we are told, the guiding role
belonged to Marxism, ``on the firm footing of democracy"? \emph{Those
parties proved bankrupt.} Those features of their previous work
which Kautsky now wishes to render permanent -- self-adaptation,
repudiation of ``illegal" activity, repudiation of the open fight,
hopes placed in democracy as the road to a painless revolution -- all
these fell into dust. In their fear of defeat, holding back the masses
from open conflict, dissolving the general strike discussions, the
parties of the Second International were preparing their own
terrifying defeat; for they were not able to move one finger to avert
the greatest catastrophe in world history, the four years' imperialist
slaughter, which foreshadowed the violent character of the civil war.
Truly, one has to put a wadded night-cap not only over one's eyes, but
over one's nose and ears, to be able to-day, after the inglorious
collapse of the Second International, after the disgraceful bankruptcy
of its leading party -- the German Social-Democracy -- after the bloody
lunacy of the world slaughter and the gigantic sweep of the civil war,
to set up in contrast to us, the profundity, the loyalty, the
peacefulness and the sobriety of the Second International, the
heritage of which we are still liquidating.






\vfill
\begin{center}
{\fontfamily{phv}\selectfont 
Modern Hacker Library\\books.modernhacker.com}
\end{center}
\end{document}
