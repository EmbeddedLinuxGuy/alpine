
\documentclass[12pt]{article}
\title{Cult of the Dark God}
\usepackage[bmargin=.25in, lmargin=.25in, rmargin=.25in, tmargin=.25in, paperwidth=4.25in, paperheight=5.5in]{geometry}
\usepackage{graphicx}
\begin{document}
\thispagestyle{empty}
\begin{center}
\textbf{\Large Cult of the Dark God}\\

\end{center}


\section{The Dark God}

Known in the North as Asmodeus the Lord of Inferno, known in the South as
Gorgoroth the Blood-Drinker, known in the East as Ahrivan the Pact-Maker, known
in the West as Belial the Soul-Eater, The Horned God is a power of the lower
plane who seeks dominion over the mortal realm. The Lord of the Inferno was
once fervently worshipped and served by a cult of powerful summoners, but after
the destruction of the shrine and stronghold at Barog Nor and the subsequent
massacre there the dark god's power in the realm is greatly diminished.

The Horned God seeks to increase his sphere of influence in the mortal realm,
not only to take revenge on the servants of Light but to magnify his power in
his own plane as well. Dark Lords and Princes of the Lower Plane have clashed
before - resecuring his foothold in the mortal world would make the Horned God
undisputed ruler of the realms below.

Disciples revere Asmodeus and hold the summoning of demons to the mortal plane
as an act of devotion and allegiance. Binding demons as unwilling servants or
familiars is disdained as petty and unwise, unbecoming of a true servant of the
Horned God. Disciples seek instead to further Asmodeus' power by rebuilding his
earthly fortress and opening the Hellgate, allowing demons to freely travel
from Inferno to the earthy plane. The greatest act of service would be securing
the Hellgate for passage for the Horned God himself.

Becoming a disciple of Gorgoroth is a dangerous path. The Horned God's
followers are scattered across the realm and live in secrecy for fear of
persecution. Most especially feared and despised by the cultists are the
Paladins of  the Just and the Clerics of the Kindly Lord, two orders of the
Light. However reviled their practices may be, there's many a village or city
with rumors of secret practicioners and loyalists to the Underthrone. It's
rumored some barons (frequently unpopular ones) dabble in forbidden lore and
bloody rituals, seeking aid from below in their disputes with rival lords and
campaigns for higher office.

\section{The Sacking of Barog Nor}

Barog Nor was the site of a fortified temple and the house of Belial on earth.
It was desecrated land where both the blood of the willing and unwilling was
spilt in act of mass sacrifice to the Horned God. The Shrine of the Red Altar
was the center of cult worship, where bloody offerings were made and forbidden
rituals performed. This was to be the site of the Hellmouth, before the
servants of the Light interviened.

The Paladins of the Just and the Clerics of the Kindly Lord, two armed templar
orders, joined in an effort of mutual self-interest to invade and sack Barog
Nor. Their aim was to destroy the Red Shrine there before a Hellmouth could be
opened, a permanent gateway to the Infernal Realm. After defeating the
Dreadguard, they slaughtered the attendant priests and warlocks - interrupting
the ritual and sealing the rift. What followers were not slain in battle fled
for their lives and are now scattered in a sinister diaspora throughout the
realm.

However, even in victory, the pious and devout are not immune to petty
squabling. After the cult's defeat, the victors disputed over the the spoils
and the division of the shrine's dark relics as each proved a proud prize for
their order. This dispute escalated into a few violent outburts between
followers of the orders. While not much blood was shed, this incident became
cause for lasting animosity between the two sacred orders. Now, meetings
between the two orders typically degenerate into heated disputes of whose order
is more devout or more righteous. Some encounters have been known to end in
violence but this is rare.

\section{Treasures of Barog Nor}

The location of the dark relics of Barog Nor is unknown. Some are likely in the
possession of the templar orders, but it's suspected that some artifacts may
have been spirited away by faithful servants of the Horned God.

\subsection{The Ebon Tome}
A codex of profane rituals and summoning spells. Warlocks and
Dark Priests recited incantations from the Ebon Tome to bring entities from the
Infernal Realm to this world. Masters of the Tome practicing from the rites
therin always summon demons loyal to Belial and never servants of another dark
god. (+3 to summoning spells beyond player's ability level. The tome guarantees
the success of the summoning or ritual but does not affect the time required to
perform it).

\subsection{The Helm of the Demonlord}
The skull of a rival demonlord who, in a foolish
act of insurrection against the Horned God, was slain in battle for dominion of
the Infernal Realm. It pleases the Horned God to look upon a disciple wearing
the skull of his former adversary. It reminds him of his victory. (The
Demonlord Helm is a heavy armor bone helm which grants the wearer the Blessing
of Damnation, a +1 Strength/+2 Charisma modifier for Chaotic).

\subsection{The Soul Prison}
A gleaming obsidian blade used by cult acolytes in ritual
sacrifice. Believed to have been forged in the Beyond, a portion of every
sacrificial soul is absorbed into the Soul Prison thus increasing its power.
The way the light skates across the black blade is unnerving to behold.

\subsection{The Lichstaff}
The skull and spinal column of a powerful necromancer and lich,
a one-time leader of the early cult-acolytes. After his exorcism at the hands
of righteous paladins, the lich's skull and spine were recovered and bound to a
staff as a powerful spell foci for necromantic spells. (+3 to all necromatic
spells beyond skill level).

\subsection{The Censor of the Damned}
Once a cleric's censor used to sprinkle holy water on
the faithful, this relic has been reforged as a morningstar and perverted by
profane rituals. When wielded in battle the Censor burns with the fire of the
raging Inferno itself. (Baptism of Fire - weapon does fire damage, +2 modifier
against Good).

\subsection{The Orb of the Unspoken Word}
Seemingly a black glass orb, it allows direct
two-way communication between the user and the Horned God, allowing his will to
be known and executed. Much like the Soul Prison, the Orb of the Unspoken Word
feeds on souls and blood sacrifice.




\vfill
\begin{center}
{\fontfamily{phv}\selectfont 
Polar Shore Press\\North Beach, California 94133}
\end{center}
\end{document}
