
\documentclass[12pt]{article}
\title{HOW NUTH WOULD HAVE PRACTISED HIS ART UPON THE GNOLES}
\usepackage[bmargin=.25in, lmargin=.25in, rmargin=.25in, tmargin=.25in, paperwidth=4.25in, paperheight=5.5in]{geometry}
\usepackage{graphicx}
\begin{document}
\thispagestyle{empty}
\begin{center}
\textbf{\Large HOW NUTH WOULD HAVE PRACTISED HIS ART UPON THE GNOLES}\\
\textbf{Lord Dunsany}\\

\end{center}

\begin{center}
\includegraphics[height=4in]{simegnoles.jpg}
\end{center}


Despite the advertisements of rival firms, it is probable that every
tradesman knows that nobody in business at the present time has a
position equal to that of Mr. Nuth. To those outside the magic circle
of business, his name is scarcely known; he does not need to
advertise, he is consummate. He is superior even to modern
competition, and, whatever claims they boast, his rivals know it. His
terms are moderate, so much cash down when the goods are
delivered, so much in blackmail afterwards. He consults your
convenience. His skill may be counted upon; I have seen a shadow on a
windy night move more noisily than Nuth, for Nuth is a burglar by
trade. Men have been known to stay in country houses and to send a
dealer afterwards to bargain for a piece of tapestry that they saw
there--some article of furniture, some picture. This is bad taste: but
those whose culture is more elegant invariably send Nuth a night or
two after their visit. He has a way with tapestry; you would scarcely
notice that the edges had been cut. And often when I see some huge,
new house full of old furniture and portraits from other ages, I say
to myself, ``These mouldering chairs, these full-length ancestors and
carved mahogany are the produce of the incomparable Nuth."

It may be urged against my use of the word incomparable that in the
burglary business the name of Slith stands paramount and alone; and of
this I am not ignorant; but Slith is a classic, and lived long ago,
and knew nothing at all of modern competition; besides which the
surprising nature of his doom has possibly cast a glamour upon Slith
that exaggerates in our eyes his undoubted merits.

It must not be thought that I am a friend of Nuth's; on the contrary
such politics as I have are on the side of Property; and he needs no
words from me, for his position is almost unique in trade, being among
the very few that do not need to advertise.

At the time that my story begins Nuth lived in a roomy house in
Belgrave Square: in his inimitable way he had made friends with the
caretaker. The place suited Nuth, and, whenever anyone came to inspect
it before purchase, the caretaker used to praise the house in the
words that Nuth had suggested. ``If it wasn't for the drains," she
would say, ``it's the finest house in London," and when they pounced on
this remark and asked questions about the drains, she would answer
them that the drains also were good, but not so good as the house.
They did not see Nuth when they went over the rooms, but Nuth was
there.

Here in a neat black dress on one spring morning came an old woman
whose bonnet was lined with red, asking for Mr. Nuth; and with her
came her large and awkward son. Mrs. Eggins, the caretaker, glanced up
the street, and then she let them in, and left them to wait in the
drawing-room amongst furniture all mysterious with sheets. For a long
while they waited, and then there was a smell of pipe-tobacco, and
there was Nuth standing quite close to them.

``Lord," said the old woman whose bonnet was lined with red, ``you did
make me start." And then she saw by his eyes that that was not the way
to speak to Mr. Nuth.

And at last Nuth spoke, and very nervously the old woman explained
that her son was a likely lad, and had been in business already but
\begin{figure}[!ht]
\begin{center}
\includegraphics[height=4.1in]{simebabies.jpg}
\end{center}
\end{figure}
wanted to better himself, and she wanted Mr. Nuth to teach him a
livelihood.

First of all Nuth wanted to see a business reference, and when he was
shown one from a jeweller with whom he happened to be hand-in-glove
the upshot of it was that he agreed to take young Tonker (for this was
the surname of the likely lad) and to make him his apprentice. And the
old woman whose bonnet was lined with red went back to her little
cottage in the country, and every evening said to her old man,
``Tonker, we must fasten the shutters of a night-time, for Tommy's a
burglar now."

The details of the likely lad's apprenticeship I do not propose to
give; for those that are in the business know those details already,
and those that are in other businesses care only for their own, while
men of leisure who have no trade at all would fail to appreciate the
gradual degrees by which Tommy Tonker came first to cross bare boards,
covered with little obstacles in the dark, without making any sound,
and then to go silently up creaky stairs, and then to open doors, and
lastly to climb.

Let it suffice that the business prospered greatly, while glowing
reports of Tommy Tonker's progress were sent from time to time to the
old woman whose bonnet was lined with red in the labourious
handwriting of Nuth. Nuth had given up lessons in writing very early,
for he seemed to have some prejudice against forgery, and therefore
considered writing a waste of time. And then there came the
transaction with Lord Castlenorman at his Surrey residence. Nuth
selected a Saturday night, for it chanced that Saturday was observed
as Sabbath in the family of Lord Castlenorman, and by eleven o'clock
the whole house was quiet. Five minutes before midnight Tommy Tonker,
instructed by Mr. Nuth, who waited outside, came away with one
pocketful of rings and shirt-studs. It was quite a light pocketful,
but the jewellers in Paris could not match it without sending
specially to Africa, so that Lord Castlenorman had to borrow bone
shirt-studs.

Not even rumour whispered the name of Nuth. Were I to say that this
turned his head, there are those to whom the assertion would give
pain, for his associates hold that his astute judgment was unaffected
by circumstance. I will say, therefore, that it spurred his genius to
plan what no burglar had ever planned before. It was nothing less than
to burgle the house of the gnoles. And this that abstemious man
unfolded to Tonker over a cup of tea. Had Tonker not been nearly
insane with pride over their recent transaction, and had he not been
blinded by a veneration for Nuth, he would have--but I cry over spilt
milk. He expostulated respectfully; he said he would rather not go; he
said it was not fair; he allowed himself to argue; and in the end, one
windy October morning with a menace in the air found him and Nuth
drawing near to the dreadful wood.

Nuth, by weighing little emeralds against pieces of common rock, had
ascertained the probable weight of those house-ornaments that the
gnoles are believed to possess in the narrow, lofty house wherein they
have dwelt from of old. They decided to steal two emeralds and to
carry them between them on a cloak; but if they should be too heavy
one must be dropped at once. Nuth warned young Tonker against greed,
\begin{figure}[!ht]
\begin{center}
\includegraphics[height=4.1in]{simewolves.jpg}
\end{center}
\end{figure}
and explained that the emeralds were worth less than cheese until they
were safe away from the dreadful wood.

Everything had been planned, and they walked now in silence.

No track led up to the sinister gloom of the trees, either of men or
cattle; not even a poacher had been there snaring elves for over a
hundred years. You did not trespass twice in the dells of the gnoles.
And, apart from the things that were done there, the trees themselves
were a warning, and did not wear the wholesome look of those that we
plant ourselves.

The nearest village was some miles away with the backs of all its
houses turned to the wood, and without one window at all facing in
that direction. They did not speak of it there, and elsewhere it is
unheard of.

Into this wood stepped Nuth and Tommy Tonker. They had no firearms.
Tonker had asked for a pistol, but Nuth replied that the sound of a
shot ``would bring everything down on us," and no more was said about
it.

Into the wood they went all day, deeper and deeper. They saw the
skeleton of some early Georgian poacher nailed to a door in an oak
tree; sometimes they saw a fairy scuttle away from them; once Tonker
stepped heavily on a hard, dry stick, after which they both lay still
for twenty minutes. And the sunset flared full of omens through the
tree trunks, and night fell, and they came by fitful starlight, as
Nuth had foreseen, to that lean, high house where the gnoles so
secretly dwelt.

All was so silent by that unvalued house that the faded courage of
Tonker flickered up, but to Nuth's experienced sense it seemed too
silent; and all the while there was that look in the sky that was
worse than a spoken doom, so that Nuth, as is often the case when men
are in doubt, had leisure to fear the worst. Nevertheless he did not
abandon the business, but sent the likely lad with the instruments of
his trade by means of the ladder to the old green casement. And the
moment that Tonker touched the withered boards, the silence that,
though ominous, was earthly, became unearthly like the touch of a
ghoul. And Tonker heard his breath offending against that silence, and
his heart was like mad drums in a night attack, and a string of one of
his sandals went tap on a rung of a ladder, and the leaves of the
forest were mute, and the breeze of the night was still; and Tonker
prayed that a mouse or a mole might make any noise at all, but not a
creature stirred, even Nuth was still. And then and there, while yet
he was undiscovered, the likely lad made up his mind, as he should
have done long before, to leave those colossal emeralds where they
were and have nothing further to do with the lean, high house of the
gnoles, but to quit this sinister wood in the nick of time and retire
from business at once and buy a place in the country. Then he
descended softly and beckoned to Nuth. But the gnoles had watched him
through knavish holes that they bore in trunks of the trees, and the
unearthly silence gave way, as it were with a grace, to the rapid
screams of Tonker as they picked him up from behind--screams that came
faster and faster until they were incoherent. And where they took him
it is not good to ask, and what they did with him I shall not say.

Nuth looked on for a while from the corner of the house with a mild
surprise on his face as he rubbed his chin, for the trick of the holes
in the trees was new to him; then he stole nimbly away through the
dreadful wood.

``And did they catch Nuth?" you ask me, gentle reader.

``Oh, no, my child" (for such a question is childish). ``Nobody ever
catches Nuth."




\vfill
\begin{center}
{\fontfamily{phv}\selectfont 
Modern Hacker Library\\books.modernhacker.com}
\end{center}
\end{document}
